\documentclass[sigconf]{acmart}

\input{format/final}

\begin{document}
\title{Modeling Opioid Pain Reliever Misuse and Abuse}
  \author{Sean M. Shiverick}
  \affiliation{
  \institution{Indiana University-Bloomington}
  }
\email{smshiver@iu.edu}

\renewcommand{\shortauthors}{S.M. Shiverick}

%%%%%%%%%%%%%%%%%%%%%%%%%%%%%%%%%%%%%%%%%%%%%%%%%%%%%%%%%%%%%%%%%%%%%%%%%%%%%%%%

\begin{abstract}
  
The misuse and abuse of prescription opioids (MUPO) is influenced by both 
personal characteristics and environmental factors. Aggregated measures of 
pain reliever use, misuse, and abuse, demographic factors, medications, and 
illicit drugs, were constructed using data from the National Survey on Drug 
Use and Health (NSDUH 2015-16). Weighted Least Squares (WLS) regression of 
pain reliever misuse and abuse was conducted with a subset of \textit{N} = 
6,946 individuals who self-reported MUPO. Age, sex, marital status, and 
education level were negatively related to pain reliever misuse and abuse,
whereas use of medications and illicit drugs were positively related to 
PRLMISAB. Overall, individuals who were older, female, married, with a higher
level of education were less likely to misuse and abuse opioid pain relievers 
than their counterparts. The effects of demographic factors varied by age group. 
Past use of any opioid pain relievers or tranquilizers significantly predicted 
PRLMISAB across all age groups. Overall, the proportion of PRLMISAB was higher 
among individuals who had used heroin than those who had not; however, use of 
cocaine and amphetamines were more influential predictors of PRLMISAB than 
heroin. The relation between illicit drug use and PRLMISAB also varied by age 
group. Heroin was a significant predictor of PRLMISAB between 18 to 49 years; 
use of amphetamines significantly predicted PRLMISAB between 12 to 35 years. 
These findings suggest that some demographic characteristics could be 
preventative of opioid misuse, and may contribute to resilience in overcoming 
opioid abuse and addiction. \footnote{\textit{Sean Shiverick} is with the 
School of Informatics and Computing. This project was completed in partial 
fulfillment of requirements for the Master's in Data Science program at 
IU-Bloomington in May, 2018. Address correspondence to: smshiver@iu.edu}

\end{abstract}

\keywords{Opioid Misuse, Pain Reliever Abuse, Linear Regression Models}

\maketitle

%%%%%%%%%%%%%%%%%%%%%%%%%%%%%%%%%%%%%%%%%%%%%%%%%%%%%%%%%%%%%%%%%%%%%%%%%%%%%%%%

\section{Introduction}

Over the past two decades, prescription opioid misuse, abuse, and addiction 
in the U.S. has become a major crisis with serious public health consequences.
In 2015, an estimated 2 million Americans suffered from substance use disorders 
related to opioid pain medications such as oxycodone and hydrocodone 
\cite{nida18,cdc18}. Of patients legitimately prescribed opioids for chronic 
pain, approximately 25\% misused them, between 8\% to 12\% became addicted, and 
4\% to 6\% transitioned to heroin \cite{vowles15, carlson16}. Opioid dependence 
and addiction are chronic health conditions; following treatment, many addicted 
individuals are at high risk for relapse and overdose death \cite{shaham03}. 
Since 1999, the number of overdose deaths from prescription opioids has more 
than quadrupled \cite{cdc16}. The economic cost of the opioid crisis since 2001 
is estimated to exceed one trillion dollars \cite{altarum18}. 

Past studies have identified gender, ethnicity, psychological disorders, and 
non-opioid drug abuse as risk factors for opioid abuse \cite{yokell13,rice12}. 
Rates of hospitalization for prescription opiate overdose (POD) are higher for 
females than males, and the increase in POD is highest for Whites compared to 
Blacks or Hispanics \cite{unick13}. Supply-based interventions to reduce the 
availability of prescription opioids have produced a shift to heroin use, and 
the exponential increase in POD and heroin overdose deaths (HOD) are correlated 
\cite{jones15,reifler12}. This study examined the relationships between 
demographic features, medications, and illicit drugs as predictors of pain 
reliever misuse and abuse. Identifying factors which are negatively correlated
with pain reliever abuse may reveal personal characteristics that prevent or
decrease opioid dependence and addiction. This approach suggests that personal 
factors may contribute to resilience in overcoming opioid abuse and addiction. 
Past research indicates that non-medical use of prescription opioids also varies 
by age \cite{mccabe12}; and models of pain reliever misuse are compared across 
age groups.

%%%%%%%%%%%%%%%%%%%%%%%%%%%%%%%%%%%%%%%%%%%%%%%%%%%%%%%%%%%%%%%%%%%%%%%%%%%%%%%%

\begin{table*}[ht]
  \caption{Summary of Variables in the NSDUH 2015-16 Aggregated Data Set}
  \label{tab:freq}
  \begin{tabular}{ll}
    \toprule
    \textit{Dependent Variable} & Label \\
    \midrule
    Prescription Opioid Pain Reliever Misuse and Abuse (0-12 Scale)& PRLMISAB  \\
    \midrule
    \textit{Demographic Variables}&   \\
    \midrule
    Age Category (1=12-17 years, 2=18-25, 3=26-34, 4=35-49, 5=50 and older)& AGECAT \\
    Biological Sex (0=Male, 1=Female)& SEX  \\
    Marital Status (0=Unmarried, 1=Divorced, 2=Widowed, 3=Married)& MARRIED  \\
    Education (1=H.S. or Less, 2=H.S. Grad., 3=Some College,  4=College Grad.)& EDUCAT  \\
    Size of City/Metropolitan Region (1=Rural, 2=Small, 3=Large)& CTYMETRO  \\
    Health Problems Aggregated  (0-10 scale)& HEALTH  \\
    Mental Health, Aggregated: adult depression, emotional distress (0-10 scale)& MENTHLTH  \\
    Treatment for Drugs and Alcohol in past year, Aggregated (0-5 scale)& TRTMENT  \\
    Mental Health Treatment, Aggregated (Likert scale, 1-10)& MHTRTMT  \\
    \midrule
    \textit{Medication and Drug Use Variables}& \\
    \midrule
    Any Prescribed Opioid Pain Reliever Use in past year, Aggregated (1-10 scale)& PRLANY  \\
    Tranquilizer use, past year, Aggregated (Likert scale, 0-5)& TRQLZRS \\
    Sedative use, past year, Aggregated (0-5 scale)& SEDATVS  \\
    Heroin use, past year, Aggregated (0-5 scale)& HEROINUSE  \\
    Cocaine and Crack Cocaine Use in past year, Aggregated  (0-5 scale)& COCAINE  \\
    Amphetamine and Methamphetamine Use in past year, Aggregated (0-5 scale)& AMPHETMN  \\
    Hallucinogen Use in past year, Aggregated (0-5 scale)& HALUCNG \\
    \bottomrule
  \end{tabular}
\end{table*}


%%%%%%%%%%%%%%%%%%%%%%%%%%%%%%%%%%%%%%%%%%%%%%%%%%%%%%%%%%%%%%%%%%%%%%%%%%%%%%%%

Many researchers have analyzed the probability that an individual abused 
or misused opioids (or did not) with logistic regression 
\cite{rice12, unick13, jones15, mccabe12}. The logistic model takes the form:
$ Pr(abuse=1) = F(\alpha+\sum_j(\beta_i*X_ij)) $, where the dependent variable
is a binary outcome. In the present study, pain reliever misuse and abuse was 
assessed along a continuum by aggregating responses for several related binary 
responses into a single measure. Similarly, responses for related independent 
measures were summed, creating several aggregated variables as regressors in a 
linear regression model rather than a logit model. The general linear model 
(Ordinary Least Squares) estimates the coefficients of the independent 
variables in relation to the dependent variable, with some degree of error. 

\begin{equation}
  \ Y_i = \beta_1 + \beta_2X_2 +\beta_3X_3 +... + \beta_iX_ij + u_i
\end{equation}

%%%%%%%%%%%%%%%%%%%%%%%%%%%%%%%%%%%%%%%%%%%%%%%%%%%%%%%%%%%%%%%%%%%%%%%%%%%%%%%%

\subsection{Generalized Least Squares Model}

The classical linear regression model is based on several assumptions, 
including: the model is linear in the parameters, the fixed x-values are 
independent of the error term, the mean value of the error terms is zero, 
non-collinearity among the independent variables, and independent 
observations (i.e., non-autocorrelation). An important assumption for the 
present study, is that the disturbances $u_i$ all have the same variances 
(i.e., \emph{homoscedasticity}) \citeN{gujarati09}. Heteroscedastic 
(\emph{non-equal}) variances is a common occurrence in socio-economic 
variables measured in cross-section, but can also arise due to skewness in 
the distribution of one or more regressors in the model, or from incorrect 
functional form (e.g., linear versus log-linear models). Outliers can also 
contribute to heteroscedasticity. If the assumption of homoscedasticity is 
not met, the ordinary least squares (OLS) parameter estimates (though still 
linear and unbiased) are no longer ``best'', are not efficient as they do 
not provide the minimum variance, and yield unreliable statistics. 

In the presence of heteroscedasticity, Weighted Least Squares (WLS)-a subset 
of Generalized Least Squares (GLS)-provides estimates with a smaller variance 
to replace the OLS estimators.  To control for heteroscedasticity using WLS, 
first the primary contributing variable(s) must be identified; then an 
appropriate transformation is carried out on the data to correct for 
heteroscedasicity. OLS is then applied to the transformed data. WLS provides 
estimates which are ``best'', ``linear'', ``unbiased'' (i.e., \textit{BLUE}). 
A common ad hoc assumption with WLS is that the error variance is proportional 
to the square of one of the explanatory variables: $\sigma^2=(\sigma^2(X_ij^2))$ 
\cite{gujarati09}. The original model is transformed by dividing through 
by the explanatory variable $X_i$:

\begin{equation}
  \ Y_i/X_2 = \beta_1/X_2 + \beta_2 +\beta_3X_3/X_2 +... + \beta_iX_ij/X_2 + u_i/X_2
\end{equation}

%%%%%%%%%%%%%%%%%%%%%%%%%%%%%%%%%%%%%%%%%%%%%%%%%%%%%%%%%%%%%%%%%%%%%%%%%%%%%%%%

\subsection{National Survey on Drug Use and Health} 

Data for the study was derived from the National Survey on Drug Use and Health 
(NSDUH) from 2015 and 2016$-$obtained from the Substance Abuse and Mental Health 
Data Archive (SAMHDA) \cite{samhsa16}. The NSDUH consists of over 2600 variables 
covering all aspects of substance use, misuse, dependency, abuse, and addiction 
for tobacco, alcohol, medications and illicit drugs, including comprehensive
demographic information, including health, mental health, substance treatment
and counseling. According to the NSDUH codebook, sampling was weighted across 
states by population size for a representative distribution selected from 6,000 
area segments. The sample design used five state sample size groups, drawing more 
heavily from the eight states with the largest population (e.g., CA, FL, IL, MI, 
NY, OH, PA, TX) which together account for 48 percent of the total U.S. population 
aged 12 or older. All identifying information in the public data set were 
collapsed (e.g., age categories) and state identifiers were removed from the public 
use file to ensure confidentiality. The NSDUH public-use files do not include 
geographic location, or demographic variables related to ethnicity or immigration 
status. For the purposes of the present study, approximately 90 variables from the 
NSDUH were selected and aggregated to construct a data set of 20 variables. 

\begin{table*}[ht]
  \caption{Summary Statistics for Aggregated Variables for Full Sample and 
  Subset of Individuals Reporting Misuse and Abuse of Prescription Opioids (MUPO)}
  \label{tab:freq}
  \begin{tabular}{llllllll}
    \toprule
     & Full Sample& (N = 25,432)&&& MUPO Subset& (N = 6,946)&  \\
    \midrule
    Variables & Mean& S.D.& Median& Range& Mean& S.D.& Median  \\
    \midrule
    PRL Misuse Abuse& 1.74& 3.30& 0& 12& 6.38& 3.11 & 7 \\
    Health Problems& 2.77& 1.15& 3& 7& 2.71& 1.11& 3  \\
    Mental Health& 1.57& 2.37& 0& 12& 2.21& 2.71& 1  \\
    Drug Treatment& 0.15& 0.89& 0& 10& 0.43& 1.48& 0  \\
    MH Treatment& 0.37& 0.86& 0& 7& 0.48& 0.97& 0  \\
    Any PRL Use& 1.61& 1.06& 1& 10& 1.93& 1.37& 1 \\
    Tranquilizer Use& 0.60& 1.22& 0& 5& 1.05& 1.42& 0  \\
    Sedative Use& 0.12& 0.50& 0& 5& 0.18& 0.62& 0  \\
    Heroin Use& 0.10& 0.49& 0& 5& 0.30& 0.85& 0  \\
    Cocaine use& 0.38& 0.80& 0& 5& 0.83& 1.10& 0 \\
    Amphetamines& 0.32& 0.72& 0& 5& 0.73& 1.04& 0  \\
    Hallucinogens& 0.65& 1.16& 0& 5& 1.42& 1.47& 1  \\
    \bottomrule
  \end{tabular}
\end{table*}

%%%%%%%%%%%%%%%%%%%%%%%%%%%%%%%%%%%%%%%%%%%%%%%%%%%%%%%%%%%%%%%%%%%%%%%%%%%%%%%%
\subsection{Study Goals} 

The main goal was to assess demographic and environmental features as 
predictors of pain reliever misuse and abuse. It was hypothesized that 
demographic variables such as marital status and education level may play a 
preventive role associated with decreased pain reliever misuse and abuse. 
Based on past findings that misuse of prescription opioids varies by age, 
age group was an important covariant in the model, which was believed to be a
primary contributor to heteroscedasticity; WLS transformations were used to 
correct for heteroscedasticity due to age category. Pain reliever misuse and 
abuse was expected to decrease with age. Use of non-opiod medications and 
illicit drugs was predicted to be positively associated with pain reliever 
misuse; heroin use was expected to be highly correlated with pain reliever 
misuse and abuse. The analysis also examined the subset of individuals who 
reported ever misusing or abusing pain relievers in the past. Finally, 
separate regression models were constructed for each age group to examine 
differences in the relative influence of predictors across different subsets. 

%%%%%%%%%%%%%%%%%%%%%%%%%%%%%%%%%%%%%%%%%%%%%%%%%%%%%%%%%%%%%%%%%%%%%%%%%%%%%%%%
\section{Method}

The following step were included in the project workflow: (1) Data cleaning and 
preparation, (3) Exploratory data analysis, (3) Data Visualization, and (4) 
Construction of regression models for opioid pain reliever misuse and abuse. 

\subsection{Data Cleaning and Preparation }

The data was extracted and prepared using an interactive Python notebook 
\cite{mckinney17}. NSDUH data files for 2015 and 2016 were downloaded from 
the SAMHDA website \cite{samhsa16} URL, and saved as data frame objects in 
Pandas. The initial data files consisted of 57,146 observations from 2015, 
and 56,897 observations for 2016, with over 2,600 features. The data was 
subset by columns, selecting approximately 90 variables that included 
demographic information, health, mental health, medication use, illicit drug 
use, and treatment for drugs or alcohol or mental health. The following steps 
were taken to detect and remove inconsistencies in the data: (a) Remove missing 
values (i.e., NaN); (b) Recode blanks, non-responses, or legitimate skips 
(e.g., 99, 991, 993) to zero; (c) Recode dichotomous responses (Yes=1 / 
No=0) so that No=0; (d) Recode categorical variables to be consistent with 
amount or degree (1=low, 2=med, 3=high); (e) Rename selected variables for 
better description (e.g., Major Depressive Episode Lifetime = DEPMELT).

\subsubsection{Aggregated Variables} 

Responses for related binomial variables were summed to create aggregated 
variables; Table 1 lists the initial set of variables and range of scores. 
For example, a measure of overall health was created by combining any previous 
health problems (e.g., STDs, Hepatitis, HIV, Cancer, Hospitalization). Overall 
mental health was assessed as a combination of any adult major depressive 
episode, severe emotional distress, suicidal thoughts, or plans. The use of any 
prescription opioid pain reliever medications taken in past year (`PRLANY`) was 
assessed by summing ten of the most commonly used opioids (e.g., Hydrocodone, 
Oxycodone, Tramadol, Morphine, Fentanyl, Oxymorphone, Demerol, Hydromorphone). 
The main dependent variable, pain reliever misuse and abuse (PRLMISAB), was 
based on the self-reported non-medical use of prescription opioids not directed 
by a physician, use of pain relievers in amounts greater than prescribed, misuse 
of pain relievers in the past year or month, dependence or abuse of pain 
relievers, and use of pain relievers to ``get high''. Similar aggregated measures 
were constructed for the use and misuse of prescription Tranquilizers, Sedatives, 
Heroin (HEROINUSE), Cocaine, Amphetamines, and Hallucinogens. Drug or alcohol 
treatment was summed across different treatment contexts (e.g., Inpatient, 
Outpatient, Hospital, MH Clinic, ER, Drug Treatment Status). The aggregated 
variables were described in detail in a data codebook \citeN{shiverick18}. 

%%%%%%%%%%%%%%%%%%%%%%%%%%%%%%%%%%%%%%%%%%%%%%%%%%%%%%%%%%%%%%%%%%%%%%%%%%%%%%%%%

\subsubsection{Data Subsets} 
 
The data files were subset to select only respondents who indicated using
any prescription opioid pain reliever; 13,916 individuals (24.35\%) from 2015 
and 11,690 individuals (20.54\%) from 2016 were combined into a single file, 
providing a pooled, cross-sectional sample of N=25,606 observations. Preliminary 
examination of the data revealed 174 outliers that exceeded the range of the 
dependent variable and were excluded. The revised sample consisting of N=25,432 
observations (10655 male, 14777 female) with 17 variables was exported to CSV 
file. An additional data subset consisted of N=6,964 individuals who reported 
misusing or abusing prescription opioids (i.e., MUPO) at some point. 

\begin{figure}[!ht]
  \centering\includegraphics[width=\columnwidth]{images/Figure1.pdf}
  \caption{Proportion of Individuals Reporting Opioid Pain Reliever Misuse 
  or Abuse (PRLMISAB) by Age Group}
  \label{f:Figure1}
\end{figure}

%%%%%%%%%%%%%%%%%%%%%%%%%%%%%%%%%%%%%%%%%%%%%%%%%%%%%%%%%%%%%%%%%%%%%%%%%%%%%%%%
\section{Results}

\subsection{Exploratory Data Analysis}

Overall, the majority of individuals reported never misusing or abusing 
prescription opioids, but 27.25\% of individuals reported pain reliever misuse
and abuse. Figure 1 shows that the proportion of pain reliever misuse 
or abuse was highest for individuals between 18 to 25 years, and decreased
with age. In general, pain reliever misuse and abuse was higher among younger 
than older age groups. More women (61\%) than men (39\%) reported using any
prescription opioid pain relievers, but equal proportions of men and women 
(50\%) reported misuse and abuse of opioid pain relievers. Table 2 provides 
the summary statistics of aggregated variables measured on a scale for both 
the full sample and the MUPO subset. The mean and standard deviation for the 
dependent variable are more robust in the MUPO subset than the full sample.  

\begin{figure}[!ht]
  \centering\includegraphics[width=\columnwidth]{images/Figure2.pdf}
  \caption{Proportion of Individuals Reporting Pain Reliever Misuse
  and Abuse and Heroin Use (N=25,432}
  \label{f:Figure2}
\end{figure}

%%%%%%%%%%%%%%%%%%%%%%%%%%%%%%%%%%%%%%%%%%%%%%%%%%%%%%%%%%%%%%%%%%%%%%%%%%%%%%%%

Figure 2 shows the proportion of individuals who reported misusing prescription 
opioid pain relievers and also using heroin. The left column of the Figure 1 
shows the majority of respondents (89 percent) stated they had never misused 
opioid pain medication or used heroin, although 10 percent reported misusing 
opioid pain medication at some point. The right panel of Figure 1 shows that, 
of those individuals who reported using heroin, the proportion who reported 
misusing opioid pain medication was roughly twice as large as the proportion 
who reported only using heroin. This is consistent with the hypothesized 
link misuse of prescription opioids and heroin use. 

\begin{figure}[!ht]
  \centering\includegraphics[width=\columnwidth]{images/Figure3.pdf}
  \caption{Plot of Residuals versus Predicted Values of the Dependent Variable: 
  Pain Reliever Misuse and Abuse (N=25,432)}
  \label{f:Figure3}
\end{figure}

%%%%%%%%%%%%%%%%%%%%%%%%%%%%%%%%%%%%%%%%%%%%%%%%%%%%%%%%%%%%%%%%%%%%%%%%%%%%%%%%

Although relatively few individuals in the sample reported ever using heroin 
(4.7\% overall, 692 males, 505 females), Figure 2 shows that the proportion of 
pain reliever misuse and abuse was higher for individuals who had used heroin 
than for those who had not, which is consistent with the hypothesized link 
between misuse of prescription opioids and heroin use. (Note: some predictor 
variables used for EDA were excluded from regression analysis;
e.g, PRLMISEVR, HEROINEVR). 

\begin{table*}[ht]
  \caption{Weighted Least Squares (WLS) Parameter Estimates for Regression 
  of Pain Reliever Misuse and Abuse}
  \label{tab:freq}
  \begin{tabular}{lllllll}
    \toprule
     & Full Sample& (N = 25,432)&& MUPO Subset& (N = 6,946)&  \\
    \midrule
    Variables & Estimate& S.E.& t-value& Estimate& S.E.& t-Value  \\
    \midrule
    Intercept& 1.656***& 0.083& 19.85& 6.870***& 0.154& 44.72 \\
    Age Category& \textbf{-0.370}& 0.028& -13.23& \textbf{-0.312}***& 0.059& -5.31 \\
    Sex& \textbf{-0.156}***& 0.039& -4.03& \textbf{-0.154}*& 0.069& -2.24  \\
    Married& \textbf{-0.041}*& 0.021& -2.02& \textbf{-0.159}***& 0.039& -4.03  \\
    Education& -0.028& 0.020& -1.41& \textbf{-0.123}***& 0.037& -3.36 \\
    City/Metro Size& -0.016& 0.025& -0.67& 0.017& 0.043& 0.40 \\
    Health Problems& -0.021& 0.019& -1.12& -0.041& 0.034& -1.22 \\
    Mental Health& \textbf{0.059}***& 0.011& 5.22& -0.022& 0.018& -1.25 \\
    Drug Treatment& \textbf{0.190}***& 0.025& 7.61& -0.13& 0.024& -0.57 \\
    MH Treatment& \textbf{-0.143}***& 0.033& -4.39& 0.017& 0.029& 0.33 \\
    Any PRL Use& \textbf{0.252}***& 0.020& 12.62& \textbf{0.290}***& 0.029& 9.87 \\
    Tranquilizer Use& \textbf{0.447}***& 0.021& 21.40& \textbf{0.116}***& 0.029& 4.05 \\
    Sedative Use& \textbf{0.192}***& 0.039& 4.95& 0.043& 0.053& 0.81 \\
    Heroin Use& \textbf{0.358}***& 0.055& 6.52& \textbf{0.258}***& 0.059& 4.36 \\
    Cocaine Use& \textbf{0.350}***& 0.035& 9.95& \textbf{0.103}*& 0.043& 2.38 \\
    Amphetamines& \textbf{0.780}***& 0.030& 25.93& \textbf{0.232}***& 0.037& 6.23 \\
    Hallucinogens& \textbf{0.576}***& 0.023& 25.66& 0.006& 0.030& 0.21 \\
    \bottomrule
    Significance level:& *0.05,& **0.01& ***0.001&&&
  \end{tabular}
\end{table*}

%%%%%%%%%%%%%%%%%%%%%%%%%%%%%%%%%%%%%%%%%%%%%%%%%%%%%%%%%%%%%%%%%%%%%%%%%%%%%%%%

\subsection{Linear Regression Models}

\subsubsection{Initial OLS Regression on the Full Sample} 
 
After loading the project data into SAS, pain reliever misuse and abuse was 
regressed on the 16 independent variables using OLS with the full sample; 
this regression was statistically significant, \textit{F}(16, 24947) = 
520.47, \textit{p} $<$ 0.0001, ($R^2$ = 0.2504, adjusted $R^2$ = 0.2498). The 
variance inflation values and condition index were all within acceptable ranges, 
and the t-values for all but one independent variable in the OLS model was 
statistically significant, which indicated that near multicollinearity was 
not an issue in the data set. Examination of the residuals plotted against 
the predicted values of the dependent variable revealed an additional outlier 
that was removed. The White`s test and Goldfield-Quandt test were significant
(see Technical Appendix), which indicated heteroscedasticity in the data; 
however, the plot of the residuals versus predicted values of tbe dependent
variable, PRLMISAB in Figure 4, shows a bimodal trend in the variances that 
decreased over the predicted values. In addition, some of the residual values 
were negative. As stated in the Introduction, heteroscedasticity can occur 
due to skewness in one or more regressors included in the model. The 
Durbin-Watson test was also significant, which indicated autocorrelation 
in the data, perhaps owing to model misspecification due to excluded 
variables. 

%%%%%%%%%%%%%%%%%%%%%%%%%%%%%%%%%%%%%%%%%%%%%%%%%%%%%%%%%%%%%%%%%%%%%%%%%%%%%%%%

\subsubsection{Weighted Least Squares (WLS) Regression} 

To address heteroscedasticity, WLS transformations were applied to the data, 
based on the assumption that Age Category was the variable primarily 
contributing to heteroscedasticity. The WLS regression was conducted using 
PROC MODEL in SAS, yielded a statistically significant relationship between 
Pain Reliever Misuse and Abuse and the independent variables, 
\textit{F}(16, 24947) = 542, \textit{p} $<$ 0.0001 ($R^2$ = 0.2437, 
adjusted $R^2$ = 0.2425). The WLS transformation reduced heteroscedasticity 
from the initial OLS model, but it was not completely eliminated from the data. 
The WLS parameter estimates for the full sample given on the left side of 
Table 3 show that Education Level, Size of City/Metropolitan area, and Health 
Problems  were not significant predictors of pain reliever misuse and abuse. 
To focused on opioid pain reliever misuse and abuse, the next analysis focused 
on the  subset of misuse and abuse of prescription opioids (MUPO). 

%%%%%%%%%%%%%%%%%%%%%%%%%%%%%%%%%%%%%%%%%%%%%%%%%%%%%%%%%%%%%%%%%%%%%%%%%%%%%%%%

\subsubsection{Regression Models for the MUPO Subset} 

The following regressions were conducted on the subset of N=6,478 individuals 
who reported MUPO. The OLS regression on the MUPO subset was statistically 
significant, \textit{F}(16, 6461) = 51.69, \textit{p}$<$ 0.0001, ($R^2$ = 
0.1135, adjusted $R^2$ = 0.1113). The White’s test and Goldfield-Quandt tests 
were both statistically significant, indicating that the heteroscedasticity 
was an issue in the MUPO subset, and the OLS estimates would not be best or 
efficient. The WLS transformation was used to correct for heteroscedasticity 
due to Age Group. The WLS regression revealed a statistically significant 
relationship between Pain Reliever Misuse and Abuse and the independent 
variables, \textit{F}(16, 6462) = 51.95, \textit{p} < 0.0001 ($R^2$ = 0.1076, 
adjusted $R^2$ = 0.1056). WLS parameter for the MUPO subset are given in the 
right side of Table 3, which shows that the key demographic features were 
negatively related to pain reliever misuse and abuse, whereas medication 
use and use of illicit drugs was positively related to pain reliever misuse 
and abuse. 

\begin{table}
  \caption{WLS Parameter Estimates for Final Model of Pain Reliever Misuse and Abuse 
  with MUPO Subset (N=6964)}
  \label{tab:freq}
  \begin{tabular}{llll}
    \toprule
    Variable & Estimate& S.E.& t-Statistic \\
    \midrule
    Intercept& 6.850***& 0.098& 69.69 \\
    Age Category& \textbf{-0.327}***& 0.058& -5.64 \\
    Sex& \textbf{-0.170}*& 0.067& -4.02 \\
    Married& \textbf{-0.158}***& 0.039& -4.02 \\
    Education& \textbf{-0.130}***& 0.034& -3.81 \\
    Any PRL Use& \textbf{0.289}***& 0.029& 9.98 \\
    Tranquilizer Use& \textbf{0.114}***& 0.028& 4.14 \\
    Heroin Use& \textbf{0.244}***& 0.057& 4.24 \\
    Cocaine Use& \textbf{0.102}*& 0.04& 9.95 \\
    Amphetamines& \textbf{0.231}***& 0.036& 6.36 \\
    \bottomrule
    Significance level:& *0.05,& **0.01& ***0.001
  \end{tabular}
\end{table}

%%%%%%%%%%%%%%%%%%%%%%%%%%%%%%%%%%%%%%%%%%%%%%%%%%%%%%%%%%%%%%%%%%%%%%%%%%%%%%%%

\begin{table*}[ht]
  \caption{Comparison of Final Regression Model by Age Category for MUPO Subset
  (with Standardized Estimates)}
  \label{tab:freq}
  \begin{tabular}{llllll}
    \toprule
      & & & Age Group& &  \\
    \midrule
    Variables & 12-17& 18-25& 26-35& 36-49& 50+  \\
    \midrule
    Intercept& 6.156***& 6.147***& 5.774***& 5.227***& 5.859*** \\
           & -& -& -& -& - \\
    Sex& 0.229& \textbf{-0.474}***& \textbf{-0.701}***& -0.019& \textbf{-0.433}***  \\
           & (0.045)& \textbf{(-0.083)}& \textbf{(-0.115)}& (-0.003)& \textbf{(-0.070)} \\
    Married& 0.127& \textbf{-0.300}***& -0.081& -0.057& -0.126  \\
           & (0.022)& \textbf{(-0.096)}& (-0.037)& (-0.024)& (-0.045) \\
    Education& 0& -0.065& -0.120& \textbf{-0.159}*& \textbf{-0.352}*** \\
           &        & (-0.020)& (-0.037)& \textbf{(-0.049)}& \textbf{(-0.119)} \\
    Any PRL Use& \textbf{0.184}*& \textbf{0.390}***& \textbf{0.367}***& \textbf{0.435}***& \textbf{0.406}***  \\
           & \textbf{(0.090)}& \textbf{(0.171)}& \textbf{(0.153)}& \textbf{(0.169)}& \textbf{(0.153)} \\
    Tranquilizer Use& 0.010& \textbf{0.159}***& \textbf{0.192}**& \textbf{0.187}**& \textbf{0.242}** \\
           & (0.005)& \textbf{(0.080)}& \textbf{(0.085)}& \textbf{(0.080)}& \textbf{(0.097)} \\
    Heroin Use& 0.183& \textbf{0.196}*& \textbf{0.314}***& \textbf{0.446}**& 0.089  \\
           & (0.024)& \textbf{(0.054)}& \textbf{(0.098)}& \textbf{(0.103)}& (0.018) \\
    Cocaine Use& 0.227& 0.086& -0.073& -0.045& 0.142  \\
           & (0.064)& (0.032)& (-0.026)& (-0.015)& (0.048) \\
    Amphetamines& \textbf{0.306}**& \textbf{0.170}**& \textbf{0.201}*& 0.143& -0.083  \\
           & \textbf{(0.120)}& \textbf{(0.063)}& \textbf{(0.066)}& (0.043)& (-0.020) \\
    \midrule
    \textit{n}       & 685& 2259& 1506&  1380& 648 \\
    \textit{Percent} & 10.56& 34.87& 23.24& 21.3& 10 \\
    \textit{F-test}  & \textbf{4.50}*** & \textbf{30.52}***& \textbf{18.56}***& \textbf{15.17}***& \textbf{6.52}***  \\ 
    \textit{R-square}& 0.0444& 0.0979& 0.0902& 0.0813& 0.0755 \\ 
    \textit{Adj. R-square}& 0.0346 & 0.0947& 0.0845& 0.076& 0.0639 \\
    \bottomrule
    Significance level:&  *0.05,& **0.01& ***0.001&
  \end{tabular}
\end{table*}

%%%%%%%%%%%%%%%%%%%%%%%%%%%%%%%%%%%%%%%%%%%%%%%%%%%%%%%%%%%%%%%%%%%%%%%%%%%%%%%%

\subsection{Final Model: WLS for the MUPO subset}

The final WLS model was obtained by excluding the non-significant independent 
variables from the previous WLS regression on the MUPO subset. The final WLS 
regression demonstrated a significant relationship between Pain Reliever Misuse 
and Abuse and the independent variables, \textit{F}(16, 6462) = 51.95, 
\textit{p} $<$ 0.0001 ($R^2$ = 0.1071, adjusted $R^2$ = 0.0865). Overall, 
10.71\% of the variation in the predicted value of Pain Reliever Misuse and 
Abuse in the MUPO subset was due to changes in the independent variables; 
and 8.65\% of the variability in Pain Reliever Misuse and Abuse was accounted 
for, taking into account the number of independent variables. The parameter 
estimates for the final WLS model are given in Table 4. 

As seen in Table 4, demographic variables were negatively correlated with Pain 
Reliever Misuse and Abuse (PRLMISAM): A one-unit change in Age Category yielded a 
-0.327 decrease in the predicted value of PRLMISAB, holding the effect of other 
independent variables constant. Similarly, there was a -0.17 decrease in PRLMISAB 
for females compared to males, all other factors considered equal. A one-unit 
change in Marital Status was associated with -0.158 decrease in PRLMA, holding
constant other variables. And a one-unit increase in Educational Level was 
associated with -0.130 decrease in PRLMA, with other predictors held constant. 

Use of prescription medications and illicit drugs was positively correlated with 
Pain Reliever Misuse and Abuse in the MUPO subset. A one-unit increase in Any 
Pain Reliever Misuse or Abuse yielded a 0.289 unit increase in PRLMA, holding 
other variables constant. Likewise, a one-unit increase in Tranquilizer use was 
linked to a 0.114 unit increase in PRLMA. In terms of illicit drugs, a one-unit 
increase in Heroin use was associated with a 0.244 unit increase in PRLMA as 
expected, holding constant the effects of other independent variables. 
A one-unit increase in Cocaine use was associated with a 0.102 increase in PRLMA. 
Finally, a one-unit increase in Amphetamine use was related to a 0.231 unit 
increase in Pain Reliever Misuse and Abuse.  

%%%%%%%%%%%%%%%%%%%%%%%%%%%%%%%%%%%%%%%%%%%%%%%%%%%%%%%%%%%%%%%%%%%%%%%%%%%%%%%%

\subsubsection{Final Regression Model by Age Category for MUPO Subset}

Separate regressions were conducted for each Age Category using the set of
regressors identified as significant in the Final WLS model. Table 5 provides 
the OLS parameter estimates with standardized betas in parentheses, F-test, 
$R^2$, adjusted $R^2$, and significance level of the estimates. As shown in 
Table 5, the parameter estimates for demographic characteristics and drug use 
varied by age group, and the standardized betas (in parentheses) give 
an indication of their relative influence. In terms of demographics features, 
Sex was significantly negatively related to Pain Reliever Misuse and Abuse 
(PRLMISAB) for individuals between 18 to 36 years and over age 50, but had the
largest effect between 26 to 35 years. Marital status was negatively related 
to PRLMISAB only for individuals between 18-26, but not older individuals. 
Education level was also negatively related to PRLMISAB only for individuals 
age 36 years and higher, and this effect was greater for age 50 or older. 
Medications and drugs were positively associated with the dependent variable. 
Of the variables included in the final model (with the MUPO subset), only Any 
Pain Reliever Use and Tranquilizer Use significantly predicted PRLMISAB across 
all age groups. Heroin Use significantly predicted PRLMISAB between the ages 
of 18 to 49 years; the effect of heroin use was greatest between 36 to 49
years, but was not a significant predictor for the youngest or oldest 
respondents. Amphetamine use was a sigificant predictor of PRLMISAB between
the 12 to 35 years of age, and this effect was largest between 12 to 17 years.
For the youngest respondents, Amphetamine use was more influential as a 
predictor of PRLMISAB than Any opioid Pain Reliever Use. Finally, Cocaine was 
not a significant predictor of PRLMISAB for any age groups, although 
COCAINE was significantly related to PRLMISAB across age categories combined. 

%%%%%%%%%%%%%%%%%%%%%%%%%%%%%%%%%%%%%%%%%%%%%%%%%%%%%%%%%%%%%%%%%%%%%%%%%%%%%%%%

\section{Discussion}

The results replicated past findings that reported a rate of opioid misuse 
of 25\% among patients legitimately prescribed opioids is the U.S. Pain reliever 
misuse and abuse varied by age, with the highest rates between 18 to 25 years, 
with decreasing opioid misuse in older age groups. As predicted, misuse and 
abuse of pain relievers was also associated with heroin use, as the proportion 
of pain reliever misuse and abuse was higher for individuals who had used heroin 
than those who had not. The final model of the regression analysis revealed that, 
for respondents who self-reported misuse of pain relievers, demographic factors 
such as age, sex, marital status, and education were associated with decreased 
pain reliever misuse and abuse overall, which is consistent with hypothesis. 
The influence of demographic factors on pain reliever misuse also varied across 
age groups, with the effect of sex being larget between the ages of 26 to 36, 
the effect of marital status between 18 to 25 years, and the effect of education 
was highest for individuals age 50 or older. As predicted, the use of medications 
and illicit drugs was related to increased misuse of pain relievers. The effect 
of any previous opioid pain reliever or tranquilizer use on opioid misuse was 
consistent across ages; however, the effect of illicit drugs varied by age. 
For example, for individuals between 12 to 17, Amphetamines were more influential 
as a determinant of opioid misuse than any previous pain reliever use. Heroin use
was an influential factor in opioid misuse between the ages 18 to 36 years. 
In the final model, Cocaine use was a significant predictor of pain reliever 
misuse and abuse, but Cocaine was not significantly related to PRLMISAB in the 
regression models by age group. It is interesting to note that overall, health
problems and mental health was not related to opioid misuse, and for the subset 
of individuals who reported misusing pain relievers, neither mental health 
treatment nor drug and alcohol treatment were significant related to pain 
reliever misuse and abuse. 

%%%%%%%%%%%%%%%%%%%%%%%%%%%%%%%%%%%%%%%%%%%%%%%%%%%%%%%%%%%%%%%%%%%%%%%%%%%%%%%%
\subsection{Limitations}

In selecting the best set of variables to model pain reliever misuse and abuse, 
the resulting tradeoff was a decrease in the proportion of variance accounted 
for (i.e., $R^2$) due to a reduction in sample size and number of independent 
variables in each successive model. Although the final regression model was 
based on a smaller sample, with a lower coefficient of determination, it 
provided a parsimonious and interpretable model of opioid misuse and abuse. 
A limitation is that mis-specification due to the exclusion of relevant variables 
such as socioeconomic status, ethnicity, or region may have contributed to
heteroscedasticity and autocorrelation. In future, related proxy variables could 
be included to represent variance in the relationship of those missing variables 
with the dependent measure. A drawback of assessing pain reliever misuse and 
abuse on a scale, taken as the sum of several binary responses, is that many 
aggregated variables had bimodal distributions that were highly skewed. A 
better approach may have been to analyze the data using a logit model to obtain 
the log odds ratios of MUPO. An initial goal of the study was to analyze trends 
in prescription opioid misuse over time using panel data. Another limitation is 
that the sample included NSDUH for only two years, owing to constraints of time 
for the project, the large number of variables, and minor changes in format 
between survey versions. Given additional time and resources, it would be 
preferable to obtain data across multiple years, covering a five- or ten-year 
period, using a more inclusive set of NSDUH variables to construct more 
informative models of opioid misuse ana abuse.

%%%%%%%%%%%%%%%%%%%%%%%%%%%%%%%%%%%%%%%%%%%%%%%%%%%%%%%%%%%%%%%%%%%%%%%%%%%%%%%%
\section{Conclusion}

In conclusion, demographic information, medication use, and illicit drug use 
play different roles as determinants of prescription opioid misuse. The current
findings suggest that personal characteristics may play a preventative role 
in decreasing pain reliever misuse and abuse. Overall, respondents who were 
older, female, married, with a higher level of education, were less likely to 
misuse and abuse opioid pain relievers than their counterparts. By contrast, 
any previous prescription opioid use, tranquilizer use, heroin, cocaine, or
amphetamine use put people at increased risk for opioid misuse and abuse. It is 
possible that the widespread practice of prescribing opioids for chronic pain 
is a primary factor driving the rapid escalation of opioid addiction and abuse. 
On the other hand, the widespread misuse and abuse of opioids may be 
influenced by mood regulation, and maintained by physical dependence. Theories 
of addiction suggest that situational cues or events can elicit a motivational 
state underlying relapse, as addictive behavior are reinstated by exposure to 
drugs, drug-related cues, or environmental stressors \cite{shaham03}. A lack of 
continuity in treatment can leave many people in recovery at risk for relapse 
or possible overdose as they return to the same environments and social 
relationships associated with their drug use. The sharp increase in overdose 
deaths in the U.S. due to synthetic opioids (other than methadone) has coincided 
with the increased availability of illicitly manufactured fentanyl \cite{nida17}. 
Because the dosage levels and potency of illicit opioids are largely unknown, 
there is greater risk of drug overdose death. The present study may help
raise awareness of personal characteristics that may help individuals overcome 
or resist opioid misuse and dependency. Additional research is needed to better 
understand characteristics associated with resilience in the face or opioid 
abuse and addiction. 


%%%%%%%%%%%%%%%%%%%%%%%%%%%%%%%%%%%%%%%%%%%%%%%%%%%%%%%%%%%%%%%%%%%%%%%%%%%%%%%%

\begin{acks}

Portions of this paper were completed as part of a course project in Data 
Analysis and Modeling taught by Dr. Barry Rubin (SPEA Connect P507) at 
Indiana University in Spring 2018. The author would like to thank Professor 
Rubin and Dr. Venkat Nadella the TA for helpfulf guidance and feedback on 
the project proposal and analysis. Thanks to Taylor Roundtree for conducting
tests and corrections for heteroscedasticity; Abhishek Babuji completed 
the tests and correction for autocorrelation.

\end{acks}
%ACM-Reference-Format
\bibliographystyle{unsrt}
\bibliography{report} 

%%%%%%%%%%%%%%%%%%%%%%%%%%%%%%%%%%%%%%%%%%%%%%%%%%%%%%%%%%%%%%%%%%%%%%%%%%%%%%%%

\appendix

\section{Technical Appendix}

Note: Tests of the OLS assumptions reported below were conducted using 
a data subset with additional outliers removed, and therefore the sample 
\textit{N}'s reported in this section are different from the sample sizes 
reported in the body of the paper.

\subsection{Detection and Correction of Heteroscedasticity}

There was some concern for heteroscedasticity in the data because of the 
demographic variables measured in cross-section, high degree of skew in the 
aggregated variables, presence of outliers, and non-normal distributions of 
errors. An auxiliary regression conducted for the White`s test yielded a 
$R^2$ = 0.096, with N = 24,963 and 121 parameters, which provided a $\chi^2$
= 2401.44, that was greater than the critical value of $\chi^2$ = 124.34 
(df=100), at the $\alpha$ = 0.05; the null hypothesis was rejected, 
indicating evidence of strong heteroscedasticity in the data.  

The Goldfield-Quandt Test was conducted using \textit{C} = 2952 (16 parameters), 
which resulted in a significant F-test result of 1.24 which was greater than 
the critical value of \textit{F}(500,200) = 1.22, at the $\alpha$ = 0.05 level; 
the null hypothesis was rejected providing evidence of heteroscedasticity. 
To correct for heteroscedasticity, the Weighted Least Squares (WLS) 
transformation was used based on the assumption that $\alpha^2$ = 
$\alpha^2((AGECAT)^2)$, with PROC MODEL in SAS to obtain the WLS regression 
equation and White`s test for this new model. The White`s test from PROC 
MODEL WLS regression corrected for heteroscedasticity yielded a $\chi^2$ = 2081, 
\textit{p} $<$ 0.0001, which indicated that heteroscedasticity was reduced but 
not eliminated from the data.

The same procedure was conducted with the MUPO subset. The auxiliary 
regression for the White`s test yielded an $R^2$ = 0.064 (\textit{N} = 6478, 
121 parameters), which provided a $\chi^2$ = 414.59, that was greater than 
the critical value of $\chi^2$ = 124.34 (df=100), at the $\alpha$ = 0.05 level; 
the null hypothesis was rejected, indicating evidence of heteroscedasticity.  
The Goldfield-Quandt Test was conducted using \textit{C} = 1505 (16 parameters), 
which resulted in a \textit{F} = 1.10, which was less than the critical value 
of \textit{F}(500,200)= 1.22, at the $\alpha$ = 0.05 level; this non-significant 
result suggested there was not sufficient evidence of heteroscedasticity. 
However, the WLS transformations was used based on the assumption that 
Age Category was contributing to heteroscedasticity in the data.

%%%%%%%%%%%%%%%%%%%%%%%%%%%%%%%%%%%%%%%%%%%%%%%%%%%%%%%%%%%%%%%%%%%%%%%%%%%%%%%%

\subsection{Detection and Correction of Autocorrelation}

The Durbin-Watson statistic from the initial OLS regression, \textit{d} = 0.407, 
was less than the lower bound of \textit{d} ($d_L$ = 1.599; $d_U$ = 1.943, with 
16 independent variables, 24964 observations) indicating evidence of positive 
autocorrelation. Although the OLS estimators are still unbiased, consistent, 
and asymptotically normally distributed, they are not efficient and the residual 
variances will likely underestimate the true error variance. The value of $R^2$ 
will be overestimated, and the \textit{t} and \textit{F} statistics unreliable. 
As described by Gujarati and Porter, ``a significant \textit{d} statistic may not 
necessarily indicate autocorrelation. [...] it may be an indication of omission 
of relevant variables from the model.'' (p. 437)\cite{gujarati09}.  

The Cochrane-Orcutt correction procedure (with two iterations) was used to 
obtain the WLS parameters estimates. The first-round estimate of the 
autocorrelation coefficient \textit{rho} was 0.798 and the Durbin-Watson 
statistic, \textit{d} = 0.039, still indicated positive autocorrelation. The 
second-round estimate of the autocorrelation coefficient \textit{rho} was 0.999 
and the Durbin-Watson statistic, \textit{d} = 0.1.758, which fell in the Zone 
of Indecision; therefore, we cannot conclude that autocorrelation does or does 
not exist. The Durbin-Watson statistic for the Yule-Walker WLS regression was 1.09, 
which was in the lower bound indicating positive autocorrelation. This suggests 
that the Cochrane-Orcutt correction procedure removed more of the autocorrelation 
than the Yule-Walker WLS regression. Finally, the parameter estimates for only 
3 out of the 16 independent variables in the WLS regression for the Cochrane-Orcutt 
procedure were statistically significant. In addition, the $R^2$ = 0.038, and 
parameter estimates for only 3 independent variables were significant at the 
0.10 level in this model. 



\end{document}
