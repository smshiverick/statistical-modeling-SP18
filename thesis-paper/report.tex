\documentclass[sigconf]{acmart}

\input{format/final}

\begin{document}
\title{Contributions of Demographic Features, Medications, and Illicit Drugs 
to Opioid Pain Reliever Misuse and Abuse}
  \author{Sean M. Shiverick}
  \affiliation{
  \institution{Indiana University-Bloomington}
  }
\email{smshiver@iu.edu}

\renewcommand{\shortauthors}{S.M. Shiverick}

%%%%%%%%%%%%%%%%%%%%%%%%%%%%%%%%%%%%%%%%%%%%%%%%%%%%%%%%%%%%%%%%%%%%%%%%%%%%%%%%

\begin{abstract}
  
The misuse and abuse of prescription opioids (MUPO) is influenced by both 
personal and environmental factors. Aggregated measures of pain reliever use, 
misuse, and abuse, demographic factors, use of medications, and illicit drugs, 
were constructed using data from the National Survey on Drug Use and Health 
(NSDUH 2015-2016). Weighted Least Squares (WLS) regression of pain reliever 
misuse and abuse (PRLMISAB) was conducted with a subset of \textit{N} = 6,946 
individuals who self-reported MUPO. Age, sex, marital status, and education 
level were negatively related to PRLMISAB, whereas use of medications and 
illicit drugs were positively associated with PRLMISAB. Individuals who were 
older, female, married, with more education were less likely to misuse and 
abuse opioid pain relievers than their counterparts. The effects of these 
demographic factors varied by age group. Use of any opioid pain relievers 
predicted PRLMISAB across all age groups, followed by tranquilizer use. 
Overall, the proportion of PRLMISAB was higher among individuals who had used 
heroin than those who had not; however, cocaine and amphetamines were more 
influential predictors of PRLMISAB than heroin. The relation between illicit 
drug use and PRLMISAB also varied by age group. Heroin predicted PRLMISAB 
between 18 to 49 years; use of amphetamines significantly predicted PRLMISAB 
between 12 to 35 years. Some demographic features may play a role in
helping to prevent or decrease opioid misuse, and contribute to resilience 
in overcoming opioid abuse and addiction. \footnote{\textit{Sean Shiverick} 
is with the School of Informatics and Computing. This project was completed in 
partial fulfillment of requirements for the Master's in Data Science program 
at IU-Bloomington in May, 2018. Address for correspondence: smshiver@iu.edu}

\end{abstract}

\keywords{Modeling, Opioid Misuse, Pain Reliever Abuse, Linear Regression}

\maketitle

%%%%%%%%%%%%%%%%%%%%%%%%%%%%%%%%%%%%%%%%%%%%%%%%%%%%%%%%%%%%%%%%%%%%%%%%%%%%%%%%

\section{Introduction}

Over the past two decades, prescription opioid misuse, abuse, and addiction 
in the U.S. has become a major crisis with serious public health consequences.
In 2015, an estimated 2 million Americans suffered from substance use disorders 
related to opioid pain medications such as oxycodone and hydrocodone 
\cite{nida18,cdc18}. Of patients legitimately prescribed opioids for chronic 
pain, approximately 25\% misused them, between 8\% to 12\% became addicted, and 
4\% to 6\% transitioned to heroin \cite{vowles15, carlson16}. Opioid dependence 
and addiction are chronic health conditions; following treatment, many addicted 
individuals are at high risk for relapse and overdose death \cite{shaham03}. 
Since 1999, the number of overdose deaths from prescription opioids has more 
than quadrupled \cite{cdc16}. The economic cost of the opioid crisis since 2001 
is estimated to exceed one trillion dollars \cite{altarum18}. 

Previous studies have identified several risk factors for opioid abuse, 
including, but not limited to: gender, ethnicity, comorbid psychological 
disorders, and non-opioid drug abuse \cite{yokell13,rice12,zedler14}. Rates of 
hospitalization for prescription opiate overdose (POD) are higher for females 
than males, and the increase in POD is highest for Whites compared to Blacks or 
Hispanics \cite{unick13}. Supply-based interventions to reduce the availability 
of prescription opioids have produced a shift to heroin use, and the exponential 
increase in POD and heroin overdose deaths (HOD) are correlated 
\cite{jones15,reifler12}. The current study examined the contributions of 
demographic features, use of medications, and illicit drugs as predictors of 
pain reliever misuse and abuse. It was hypothesized that personal 
characteristics may play a role in resisting or overcoming opioid dependence 
and addiction. According to this view, some individuals may be more resilient 
than others in recovering from opioid abuse and addiction. Insofar as the 
non-medical use of prescription opioids also varies by age \cite{mccabe12}, 
models of pain reliever misuse and abuse were compared across age groups.

%%%%%%%%%%%%%%%%%%%%%%%%%%%%%%%%%%%%%%%%%%%%%%%%%%%%%%%%%%%%%%%%%%%%%%%%%%%%%%%%

\begin{table*}[ht]
  \caption{Summary of Variables in the NSDUH 2015-16 Aggregated Data Set}
  \label{tab:freq}
  \begin{tabular}{ll}
    \toprule
    \textit{Dependent Variable} & Label \\
    \midrule
    Prescription Opioid Pain Reliever Misuse and Abuse (0-12 Scale)& PRLMISAB  \\
    \midrule
    \textit{Demographic Variables}&   \\
    \midrule
    Age Category (1=12-17 years, 2=18-25, 3=26-34, 4=35-49, 5=50 and older)& AGECAT \\
    Biological Sex (0=Male, 1=Female)& SEX  \\
    Marital Status (0=Unmarried, 1=Divorced, 2=Widowed, 3=Married)& MARRIED  \\
    Education (1=H.S. or Less, 2=H.S. Grad., 3=Some College,  4=College Grad.)& EDUCAT  \\
    Size of City/Metropolitan Region (1=Rural, 2=Small, 3=Large)& CTYMETRO  \\
    Health Problems Aggregated  (0-10 scale)& HEALTH  \\
    Mental Health, Aggregated: adult depression, emotional distress (0-10 scale)& MENTHLTH  \\
    Treatment for Drugs and Alcohol in past year, Aggregated (0-5 scale)& TRTMENT  \\
    Mental Health Treatment, Aggregated (Likert scale, 1-10)& MHTRTMT  \\
    \midrule
    \textit{Medication and Drug Use Variables}& \\
    \midrule
    Any Prescribed Opioid Pain Reliever Use in past year, Aggregated (1-10 scale)& PRLANY  \\
    Tranquilizer use, past year, Aggregated (Likert scale, 0-5)& TRQLZRS \\
    Sedative use, past year, Aggregated (0-5 scale)& SEDATVS  \\
    Heroin use, past year, Aggregated (0-5 scale)& HEROINUSE  \\
    Cocaine and Crack Cocaine Use in past year, Aggregated  (0-5 scale)& COCAINE  \\
    Amphetamine and Methamphetamine Use in past year, Aggregated (0-5 scale)& AMPHETMN  \\
    Hallucinogen Use in past year, Aggregated (0-5 scale)& HALUCNG \\
    \bottomrule
  \end{tabular}
\end{table*}


%%%%%%%%%%%%%%%%%%%%%%%%%%%%%%%%%%%%%%%%%%%%%%%%%%%%%%%%%%%%%%%%%%%%%%%%%%%%%%%%

\subsection{Modeling Pain Reliever Misuse and Abuse}

Many researchers have analyzed the probability that an individual abused 
or misused opioids (or did not) with logistic regression 
\cite{rice12, unick13, jones15, mccabe12, zedler14}. The logistic model takes
a binary variable as the outcome, with the following form: $ Pr(abuse=1) = 
\textit{F}(\alpha+\sum_j(\beta_i*X_ij)) $. The function $\textit{F(z)}$
is a cumulative logistic distribution, $\textit{X}_ij$ represents a matrix 
of observation-related independent variables, $\alpha$ and $\beta_j$ are the
parameters to be estimated in the model. In the present study, pain reliever 
misuse and abuse was assessed along a continuum by aggregating responses for 
several related binary measures into a single variable. Similarly, binary 
responses for related independent measures were summed, creating several 
aggregated variables as predictors in a linear regression model, rather than 
a logit model. The general linear model (Ordinary Least Squares) estimates 
the coefficients of the independent variables in relation to the dependent 
variable, with some degree of error, in the following form: 

\begin{equation}
  \ Y_i = \beta_1 + \beta_2X_2 +\beta_3X_3 +... + \beta_iX_ij + u_i
\end{equation}

%%%%%%%%%%%%%%%%%%%%%%%%%%%%%%%%%%%%%%%%%%%%%%%%%%%%%%%%%%%%%%%%%%%%%%%%%%%%%%%%

The classical linear regression model is based on several assumptions, 
including: (a) the model is linear in the parameters, (b) the fixed x-values 
are independent of the error term, (c) the mean value of the error terms is 
zero, (d) constant variance of error terms (i.e., \emph{homoscedasticity}) 
(e) non-collinearity among the independent variables, (f) independent 
observations (i.e., non-autocorrelation), and (g) the x-values are not all 
the same \citeN{gujarati09}. An important assumption for the present study is 
that the disturbances $u_i$ all have the same variances. Heteroscedastic 
(i.e., \emph{non-equal}) variances is a common occurrence in socioeconomic 
variables measured in cross-section, but can also arise due to skewness in 
the distribution of one or more regressors in the model, or from incorrect 
functional form (e.g., linear versus log-linear models). Outliers can also 
contribute to heteroscedasticity. If the assumption of homoscedasticity is 
not met, the ordinary least squares (OLS) parameter estimates are no longer 
``best'', are not efficient as they do not provide the minimum variance, 
and produce unreliable statistics. 

In the presence of heteroscedasticity, Weighted Least Squares (WLS)$-$a subset 
of Generalized Least Squares (GLS)$-$provides estimates with a smaller variance 
to replace the OLS estimators.  To control for heteroscedasticity using WLS, 
first the primary contributing variable(s) must be identified; then an 
appropriate transformation is carried out on the data to correct for 
heteroscedasicity. OLS is then applied to the transformed data. WLS provides 
estimates which are \textit{best}, \textit{linear}, \textit{unbiased} (i.e., 
``BLUE''). A common ad hoc assumption with WLS is that the error variance is 
proportional to the square of one of the explanatory variables: 
$\sigma^2=(\sigma^2(X_ij^2))$ \cite{gujarati09}. The original model is 
transformed by dividing through by the explanatory variable $X_i$. In the 
present study, it was assumed that Age primary variable contributing to 
heteroscedasticity, and WLS transformations were used to construct the model.

\begin{equation}
  \ Y_i/X_2 = \beta_1/X_2 + \beta_2 +\beta_3X_3/X_2 +... + \beta_iX_ij/X_2 + u_i/X_2
\end{equation}

%%%%%%%%%%%%%%%%%%%%%%%%%%%%%%%%%%%%%%%%%%%%%%%%%%%%%%%%%%%%%%%%%%%%%%%%%%%%%%%%

\subsection{National Survey on Drug Use and Health} 

Data for the study was derived from the National Survey on Drug Use and Health 
(NSDUH) from 2015 and 2016$-$obtained from the Substance Abuse and Mental 
Health Data Archive (SAMHDA) \cite{samhsa16}. The NSDUH consists of over 2600 
variables covering all aspects of substance use, misuse, dependency, abuse, 
and addiction for tobacco, alcohol, medications and illicit drugs, including 
comprehensive demographic information, including health, mental health, 
substance treatment and counseling. According to the NSDUH codebook, sampling 
was weighted across states by population size for a representative distribution 
selected from 6,000 area segments. The sample design used five state sample size 
groups, drawing more heavily from the eight states with the largest population 
(e.g., CA, FL, IL, MI, NY, OH, PA, TX) which together account for 48 percent of 
the total U.S. population aged 12 or older. All identifying information in the 
public data set were collapsed (e.g., age categories) and state identifiers were 
removed from the public use file to ensure confidentiality. The NSDUH public-use 
files do not include geographic location, or demographic variables related to 
ethnicity or immigration status. For the purposes of the present study, 
approximately 90 variables from the NSDUH were selected and aggregated to 
construct a data set of 20 variables, 16 of which were used to construct the 
regression model (see Table 1). 

\begin{table*}[ht]
  \caption{Summary Statistics for Aggregated Variables for Full Sample and 
  Subset of Individuals Reporting Misuse and Abuse of Prescription Opioids (MUPO)}
  \label{tab:freq}
  \begin{tabular}{llllllll}
    \toprule
     & Full Sample& (N = 25,432)&&& MUPO Subset& (N = 6,946)&  \\
    \midrule
    Variables & Mean& S.D.& Median& Range& Mean& S.D.& Median  \\
    \midrule
    PRL Misuse Abuse& 1.74& 3.30& 0& 12& 6.38& 3.11 & 7 \\
    Health Problems& 2.77& 1.15& 3& 7& 2.71& 1.11& 3  \\
    Mental Health& 1.57& 2.37& 0& 12& 2.21& 2.71& 1  \\
    Drug Treatment& 0.15& 0.89& 0& 10& 0.43& 1.48& 0  \\
    MH Treatment& 0.37& 0.86& 0& 7& 0.48& 0.97& 0  \\
    Any PRL Use& 1.61& 1.06& 1& 10& 1.93& 1.37& 1 \\
    Tranquilizer Use& 0.60& 1.22& 0& 5& 1.05& 1.42& 0  \\
    Sedative Use& 0.12& 0.50& 0& 5& 0.18& 0.62& 0  \\
    Heroin Use& 0.10& 0.49& 0& 5& 0.30& 0.85& 0  \\
    Cocaine use& 0.38& 0.80& 0& 5& 0.83& 1.10& 0 \\
    Amphetamines& 0.32& 0.72& 0& 5& 0.73& 1.04& 0  \\
    Hallucinogens& 0.65& 1.16& 0& 5& 1.42& 1.47& 1  \\
    \bottomrule
  \end{tabular}
\end{table*}

%%%%%%%%%%%%%%%%%%%%%%%%%%%%%%%%%%%%%%%%%%%%%%%%%%%%%%%%%%%%%%%%%%%%%%%%%%%%%%%%
\subsection{Study Goals} 

The main goal of the study was to analyze the contributions of demographic 
features, medications, and illicit drugs as predictors of pain reliever misuse 
and abuse. It was hypothesized that demographic variables such as marital 
status and education level may play a preventive role in decreasing pain 
reliever misuse and abuse. As described above, past findings indicate that 
misuse of prescription opioids varies by age, and age group was considered as 
an important covariant in the model contributing to heteroscedasticity. WLS 
transformations were used to correct for heteroscedasticity due to age category. 
Pain reliever misuse and abuse was expected to decrease with age. It was 
predicted that use of non-opiod medications and illicit drugs would be 
related to increased pain reliever misuse and abuse. Specifically, 
heroin use was expected to be highly correlated with misuse and abuse of 
opioid pain relievers. Analyses were conducted using the full sample and the
subset of individuals who reported ever misusing or abusing pain relievers. 
Finally, separate regression models were constructed for each age group 
to examine differences in the relative contribution of the predictor variables 
across different subsets for each age group. 

%%%%%%%%%%%%%%%%%%%%%%%%%%%%%%%%%%%%%%%%%%%%%%%%%%%%%%%%%%%%%%%%%%%%%%%%%%%%%%%%
\section{Method}

The following step were included in the project workflow: (1) Data cleaning 
and preparation, (3) Exploratory data analysis, (3) Data Visualization, and 
(4) Construction of regression models for opioid pain reliever misuse and 
abuse. 

\subsection{Data Cleaning and Preparation }

The data was extracted, cleaned and prepared using an interactive Python 
notebook \cite{mckinney17}. NSDUH data files for 2015 and 2016 were downloaded 
from the SAMHDA website \cite{samhsa16}, and saved as data frame objects in 
Pandas. The initial data files consisted of 57,146 observations from 2015, 
and 56,897 observations for 2016, with over 2,600 features. The data was 
subset by columns, selecting approximately 90 variables that included 
demographic information, health, mental health, medication use, illicit drug 
use, and treatment for drugs or alcohol or mental health (a complete list of
variables is described in a data codebook \citeN{shiverick18}). The following 
steps were taken to detect and remove inconsistencies in the data: (a) 
Remove missing values (i.e., \textit{NaN}); (b) Recode blanks, non-responses, 
or legitimate skips (e.g., 99, 991, 993) to zero; (c) Recode dichotomous 
responses (0 = No, 1 = Yes); (d) Recode categorical variables to be consistent 
with amount or degree (1 = Low, 2 = Medium, 3 = High); (e) Rename selected 
variables for better description (e.g., DEPMELT = Major Depressive Episode 
Lifetime).

\subsubsection{Aggregated Variables} 

Responses for related binomial variables were summed to create aggregated 
variables; Table 1 lists the initial set of variables and range of scores. 
For example, a measure of overall health was created by combining any previous 
health problems (e.g., STDs, Hepatitis, HIV, Cancer, Hospitalization). Overall 
mental health was assessed as a combination of any adult major depressive 
episode, severe emotional distress, and suicidal thoughts, or plans. The use of 
any prescription opioid pain reliever medications taken in past year (PRLANY) 
was assessed by summing ten of the most commonly used opioids (e.g., Hydrocodone, 
Oxycodone, Tramadol, Morphine, Fentanyl, Oxymorphone, Demerol, Hydromorphone). 
The main dependent variable, pain reliever misuse and abuse (PRLMISAB), was 
based on the self-reported non-medical use of prescription opioids not directed 
by a physician, use of pain relievers in amounts greater than prescribed, 
misuse of pain relievers in the past year or month, dependence or abuse of 
pain relievers, and use of pain relievers to ``get high''. Similar aggregated 
measures were constructed for the use and misuse of prescription Tranquilizers, 
Sedatives, Heroin (HEROINUSE), Cocaine, Amphetamines, and Hallucinogens. Drug 
or alcohol treatment was summed across different treatment contexts (e.g., 
Inpatient, Outpatient, Hospital, MH Clinic, ER, Drug Treatment Status). 
Similarly, responses for mental health treatment or counseling were summed
across different treatment settings. 

%%%%%%%%%%%%%%%%%%%%%%%%%%%%%%%%%%%%%%%%%%%%%%%%%%%%%%%%%%%%%%%%%%%%%%%%%%%%%%%%%

\subsubsection{Data Subsets} 
 
The data files were subset to select only respondents who indicated using
any prescription opioid pain relievers; \textit{n} = 13,916 individuals 
(24.35\%) from 2015 and \textit{n} = 11,690 individuals (20.54\%) from 2016 
were combined into a single file, providing a pooled, cross-sectional sample 
of N=25,606 observations. Preliminary examination of the data revealed 174 
outliers that exceeded the range of the dependent variable and were excluded. 
The revised sample consisting of \textit{N} = 25,432 observations (10,655 male, 
14,777 female) with 17 variables was exported to CSV file. An additional data s
ubset consisted of \textit{N} = 6,964 individuals who reported misusing or 
abusing prescription opioids (i.e., MUPO). 

\begin{figure}[!ht]
  \centering\includegraphics[width=\columnwidth]{images/Figure1.pdf}
  \caption{Proportion of Individuals Reporting Opioid Pain Reliever Misuse 
  or Abuse (PRLMISAB) by Age Group}
  \label{f:Figure1}
\end{figure}

%%%%%%%%%%%%%%%%%%%%%%%%%%%%%%%%%%%%%%%%%%%%%%%%%%%%%%%%%%%%%%%%%%%%%%%%%%%%%%%%
\section{Results}

\subsection{Exploratory Data Analysis}

Overall, 27.3\% of the initial sample reported misusing or abusing 
prescription opioid pain relievers; however, the majority of individuals 
reported no pain reliever misuse or abuse. In general, pain reliever misuse 
and abuse was higher among younger age groups than older respondents. As shown 
in Figure 1, the proportion of pain reliever misuse or abuse was highest 
between 18 to 25 years, and decreased with age. More women (61\%) than men 
(39\%) reported using any prescription opioid medications, but equal 
proportions of men and women (50\%) reported pain relievers misuse and abuse. 
Table 2 provides summary statistics of the aggregated variables measured on 
a scale for both the full sample and the MUPO subset. As seen in Table 2, 
the mean and standard deviation for the dependent variable are more robust 
in the MUPO subset than the full sample.  

\begin{figure}[!ht]
  \centering\includegraphics[width=\columnwidth]{images/Figure2.pdf}
  \caption{Proportion of Individuals Reporting Pain Reliever Misuse
  and Abuse and Heroin Use (N=25,432)}
  \label{f:Figure2}
\end{figure}

%%%%%%%%%%%%%%%%%%%%%%%%%%%%%%%%%%%%%%%%%%%%%%%%%%%%%%%%%%%%%%%%%%%%%%%%%%%%%%%%

Figure 2 shows the proportion of individuals who reported ever misusing 
prescription opioid pain relievers and/or ever using heroin. The left side 
of Figure 2 shows that the majority of respondents (75\%) had never misused 
opioid pain relievers or used heroin. As described above, approximately 
one-quarter of the initial sample reported misusing opioid pain relievers at 
some point. Although relatively few individuals reported ever using heroin 
(4.7\% overall, 692 males, 505 females), the right side of Figure 2 shows 
that the proportion of individuals who had used heroin and misused pain 
relievers was more than three times greater than the proportion who reported
using heroin not not misusing opioid pain relievers. In addition, the 
proportion of pain reliever misuse and abuse for those who had ever used 
heroin was significantly greater than those than for those who had not, 
which is consistent with the hypothesis that the misuse of prescription 
opioids and heroin use are linked. (Note: some predictor variables used for
exploratory analysis were excluded from the regression models presented in 
the following section: e.g., PRLMISEVR, HEROINEVR). 

%%%%%%%%%%%%%%%%%%%%%%%%%%%%%%%%%%%%%%%%%%%%%%%%%%%%%%%%%%%%%%%%%%%%%%%%%%%%%%%%

\begin{table*}[ht]
  \caption{Weighted Least Squares (WLS) Parameter Estimates for Regression 
  of Pain Reliever Misuse and Abuse}
  \label{tab:freq}
  \begin{tabular}{lllllll}
    \toprule
     & Full Sample& (N = 25,432)&& MUPO Subset& (N = 6,946)&  \\
    \midrule
    Variables & Estimate& S.E.& t-value& Estimate& S.E.& t-Value  \\
    \midrule
    Intercept& 1.656***& 0.083& 19.85& 6.870***& 0.154& 44.72 \\
    Age Category& \textbf{-0.370}& 0.028& -13.23& \textbf{-0.312}***& 0.059& -5.31 \\
    Sex& \textbf{-0.156}***& 0.039& -4.03& \textbf{-0.154}*& 0.069& -2.24  \\
    Married& \textbf{-0.041}*& 0.021& -2.02& \textbf{-0.159}***& 0.039& -4.03  \\
    Education& -0.028& 0.020& -1.41& \textbf{-0.123}***& 0.037& -3.36 \\
    City/Metro Size& -0.016& 0.025& -0.67& 0.017& 0.043& 0.40 \\
    Health Problems& -0.021& 0.019& -1.12& -0.041& 0.034& -1.22 \\
    Mental Health& \textbf{0.059}***& 0.011& 5.22& -0.022& 0.018& -1.25 \\
    Drug Treatment& \textbf{0.190}***& 0.025& 7.61& -0.13& 0.024& -0.57 \\
    MH Treatment& \textbf{-0.143}***& 0.033& -4.39& 0.017& 0.029& 0.33 \\
    Any PRL Use& \textbf{0.252}***& 0.020& 12.62& \textbf{0.290}***& 0.029& 9.87 \\
    Tranquilizer Use& \textbf{0.447}***& 0.021& 21.40& \textbf{0.116}***& 0.029& 4.05 \\
    Sedative Use& \textbf{0.192}***& 0.039& 4.95& 0.043& 0.053& 0.81 \\
    Heroin Use& \textbf{0.358}***& 0.055& 6.52& \textbf{0.258}***& 0.059& 4.36 \\
    Cocaine Use& \textbf{0.350}***& 0.035& 9.95& \textbf{0.103}*& 0.043& 2.38 \\
    Amphetamines& \textbf{0.780}***& 0.030& 25.93& \textbf{0.232}***& 0.037& 6.23 \\
    Hallucinogens& \textbf{0.576}***& 0.023& 25.66& 0.006& 0.030& 0.21 \\
    \bottomrule
    Significance level:& *0.05,& **0.01& ***0.001&&&
  \end{tabular}
\end{table*}

%%%%%%%%%%%%%%%%%%%%%%%%%%%%%%%%%%%%%%%%%%%%%%%%%%%%%%%%%%%%%%%%%%%%%%%%%%%%%%%%

\subsection{Linear Regression Models}

\subsubsection{Initial OLS Regression on the Full Sample} 
 
All regression models were constructed in SAS. First, pain reliever misuse and 
abuse was regressed on the 16 independent variables using OLS with the full 
sample; this regression was statistically significant, \textit{F}(16, 24947) = 
520.47, \textit{p} $<$ 0.0001, ($R^2$ = 0.2504, adjusted $R^2$ = 0.2498). The 
variance inflation values and condition index were all within acceptable ranges, 
and the t-values for all but one regressor in the OLS model were significant, 
which indicated that near multicollinearity was not an issue in the data set. 
Examination of the residuals plotted against the predicted values of the 
dependent variable revealed an additional outlier that was removed (Figure 3). 
The White`s test and Goldfield-Quandt test were both significant (see Technical 
Appendix), which indicated heteroscedasticity in the data; however, the plot 
of the residuals versus predicted values of the dependent variable, PRLMISAB 
in Figure 3, shows a bimodal trend in the variances that decreased over the 
predicted values. Some of the residual values were negative. As stated above, 
heteroscedasticity can occur due to skewness in one or more regressors in the 
model. The Durbin-Watson test was also significant, which indicated 
autocorrelation in the data, perhaps owing to model misspecification due to 
excluded variables. 

\begin{figure}[!ht]
  \centering\includegraphics[width=\columnwidth]{images/Figure3.pdf}
  \caption{Plot of Residuals versus Predicted Values of the Dependent Variable: 
  Pain Reliever Misuse and Abuse (\textit{N} = 25,432)}
  \label{f:Figure3}
\end{figure}


%%%%%%%%%%%%%%%%%%%%%%%%%%%%%%%%%%%%%%%%%%%%%%%%%%%%%%%%%%%%%%%%%%%%%%%%%%%%%%%%

\subsubsection{Weighted Least Squares (WLS) Regression} 

To address heteroscedasticity, WLS transformations were applied to the 
data, based on the assumption that Age Category was primarily contributing to 
heteroscedasticity. The WLS regression was conducted using PROC MODEL in SAS, 
and yielded a statistically significant relationship between Pain Reliever 
Misuse and Abuse and the independent variables, \textit{F}(16, 24947) = 542, 
\textit{p} $<$ 0.0001 ($R^2$ = 0.2437, adjusted $R^2$ = 0.2425). The WLS 
transformation reduced heteroscedasticity from the initial OLS model, but it 
was not completely eliminated from the data. The WLS parameter estimates for 
the full sample given on the left side of Table 3 show that Education Level, 
Size of City/Metropolitan area, and Health Problems were not significant 
predictors of pain reliever misuse and abuse. To focus on prescription opioid 
misuse and abuse, the next analysis used the subset of respondents who
misused or abused prescription opioids (MUPO). 

%%%%%%%%%%%%%%%%%%%%%%%%%%%%%%%%%%%%%%%%%%%%%%%%%%%%%%%%%%%%%%%%%%%%%%%%%%%%%%%%

\subsubsection{Regression Model for the MUPO Subset} 

The following regressions were conducted with the subset of \textit{N} = 6,478 
individuals who reported MUPO. The OLS regression on the MUPO subset was 
statistically significant, \textit{F}(16, 6461) = 51.69, \textit{p} $<$ 0.0001, 
($R^2$ = 0.1135, adjusted $R^2$ = 0.1113). The White`s test and Goldfield-Quandt 
tests were both statistically significant, which indicated heteroscedasticity 
in the MUPO subset. The WLS transformation was used to correct for 
heteroscedasticity due to Age Group. The WLS regression revealed a significant 
relationship between Pain Reliever Misuse and Abuse and the independent 
variables, \textit{F}(16, 6462) = 51.95, \textit{p} < 0.0001 ($R^2$ = 0.1076, 
adjusted $R^2$ = 0.1056). The WLS parameter for the MUPO subset provided in 
the right side of Table 3 shows that the key demographic features were 
negatively related to pain reliever misuse and abuse, whereas medication 
use and use of illicit drugs was positively related to pain reliever misuse 
and abuse. 

\begin{table}
  \caption{WLS Parameter Estimates for Final Model of Pain Reliever Misuse and 
  Abuse with MUPO Subset (N=6964)}
  \label{tab:freq}
  \begin{tabular}{llll}
    \toprule
    Variable & Estimate& S.E.& t-Statistic \\
    \midrule
    Intercept& 6.850***& 0.098& 69.69 \\
    Age Category& \textbf{-0.327}***& 0.058& -5.64 \\
    Sex& \textbf{-0.170}*& 0.067& -4.02 \\
    Married& \textbf{-0.158}***& 0.039& -4.02 \\
    Education& \textbf{-0.130}***& 0.034& -3.81 \\
    Any PRL Use& \textbf{0.289}***& 0.029& 9.98 \\
    Tranquilizer Use& \textbf{0.114}***& 0.028& 4.14 \\
    Heroin Use& \textbf{0.244}***& 0.057& 4.24 \\
    Cocaine Use& \textbf{0.102}*& 0.04& 9.95 \\
    Amphetamines& \textbf{0.231}***& 0.036& 6.36 \\
    \bottomrule
    Significance level:& *0.05,& **0.01& ***0.001
  \end{tabular}
\end{table}

%%%%%%%%%%%%%%%%%%%%%%%%%%%%%%%%%%%%%%%%%%%%%%%%%%%%%%%%%%%%%%%%%%%%%%%%%%%%%%%%

\begin{table*}[ht]
  \caption{Comparison of Final Regression Model by Age Category for MUPO Subset
  (with Standardized Estimates)}
  \label{tab:freq}
  \begin{tabular}{llllll}
    \toprule
      & & & Age Group& &  \\
    \midrule
    Variables & 12-17& 18-25& 26-35& 36-49& 50+  \\
    \midrule
    Intercept& 6.156***& 6.147***& 5.774***& 5.227***& 5.859*** \\
           & -& -& -& -& - \\
    Sex& 0.229& \textbf{-0.474}***& \textbf{-0.701}***& -0.019& \textbf{-0.433}***  \\
           & (0.045)& \textbf{(-0.083)}& \textbf{(-0.115)}& (-0.003)& \textbf{(-0.070)} \\
    Married& 0.127& \textbf{-0.300}***& -0.081& -0.057& -0.126  \\
           & (0.022)& \textbf{(-0.096)}& (-0.037)& (-0.024)& (-0.045) \\
    Education& 0& -0.065& -0.120& \textbf{-0.159}*& \textbf{-0.352}*** \\
           &        & (-0.020)& (-0.037)& \textbf{(-0.049)}& \textbf{(-0.119)} \\
    Any PRL Use& \textbf{0.184}*& \textbf{0.390}***& \textbf{0.367}***& \textbf{0.435}***& \textbf{0.406}***  \\
           & \textbf{(0.090)}& \textbf{(0.171)}& \textbf{(0.153)}& \textbf{(0.169)}& \textbf{(0.153)} \\
    Tranquilizer Use& 0.010& \textbf{0.159}***& \textbf{0.192}**& \textbf{0.187}**& \textbf{0.242}** \\
           & (0.005)& \textbf{(0.080)}& \textbf{(0.085)}& \textbf{(0.080)}& \textbf{(0.097)} \\
    Heroin Use& 0.183& \textbf{0.196}*& \textbf{0.314}***& \textbf{0.446}**& 0.089  \\
           & (0.024)& \textbf{(0.054)}& \textbf{(0.098)}& \textbf{(0.103)}& (0.018) \\
    Cocaine Use& 0.227& 0.086& -0.073& -0.045& 0.142  \\
           & (0.064)& (0.032)& (-0.026)& (-0.015)& (0.048) \\
    Amphetamines& \textbf{0.306}**& \textbf{0.170}**& \textbf{0.201}*& 0.143& -0.083  \\
           & \textbf{(0.120)}& \textbf{(0.063)}& \textbf{(0.066)}& (0.043)& (-0.020) \\
    \midrule
    \textit{n}       & 685& 2259& 1506&  1380& 648 \\
    \textit{Percent} & 10.56& 34.87& 23.24& 21.3& 10 \\
    \textit{F-test}  & \textbf{4.50}*** & \textbf{30.52}***& \textbf{18.56}***& \textbf{15.17}***& \textbf{6.52}***  \\ 
    \textit{R-square}& 0.0444& 0.0979& 0.0902& 0.0813& 0.0755 \\ 
    \textit{Adj. R-square}& 0.0346 & 0.0947& 0.0845& 0.076& 0.0639 \\
    \bottomrule
    Significance level:&  *0.05,& **0.01& ***0.001&
  \end{tabular}
\end{table*}

%%%%%%%%%%%%%%%%%%%%%%%%%%%%%%%%%%%%%%%%%%%%%%%%%%%%%%%%%%%%%%%%%%%%%%%%%%%%%%%%

\subsection{Final Model: WLS for the MUPO subset}

The final model was obtained by excluding the non-significant independent 
variables from the previous WLS regression on the MUPO subset. The final WLS 
regression demonstrated a significant relationship between Pain Reliever Misuse 
and Abuse and the independent variables, \textit{F}(16, 6462) = 51.95, 
\textit{p} $<$ 0.0001 ($R^2$ = 0.1071, adjusted $R^2$ = 0.0865). Overall, 
10.71\% of the variation in the predicted value of Pain Reliever Misuse and 
Abuse in the MUPO subset was due to changes in the independent variables; 
and 8.65\% of the variability in Pain Reliever Misuse and Abuse was accounted 
for, taking into account the number of independent variables. The parameter 
estimates for the final WLS model are given in Table 4. 

As seen in Table 4, demographic variables were negatively correlated with Pain 
Reliever Misuse and Abuse (PRLMISAM): A one-unit change in Age Category yielded a 
-0.327 decrease in the predicted value of PRLMISAB, holding the effect of other 
independent variables constant. Similarly, there was a -0.17 decrease in PRLMISAB 
for females compared to males, all other factors considered equal. A one-unit 
change in Marital Status was associated with -0.158 decrease in PRLMA, holding
constant other variables. And a one-unit increase in Educational Level was 
associated with -0.130 decrease in PRLMA, with other predictors held constant. 

Use of prescription medications and illicit drugs was positively correlated with 
Pain Reliever Misuse and Abuse in the MUPO subset. A one-unit increase in Any 
Pain Reliever Misuse or Abuse yielded a 0.289 unit increase in PRLMA, holding 
other variables constant. Likewise, a one-unit increase in Tranquilizer use was 
linked to a 0.114 unit increase in PRLMA. In terms of illicit drugs, a one-unit 
increase in Heroin use was associated with a 0.244 unit increase in PRLMA, as 
expected, holding constant the effects of other independent variables. 
A one-unit increase in Cocaine use was associated with a 0.102 increase in PRLMA. 
Finally, a one-unit increase in Amphetamine use was related to a 0.231 unit 
increase in Pain Reliever Misuse and Abuse.  

%%%%%%%%%%%%%%%%%%%%%%%%%%%%%%%%%%%%%%%%%%%%%%%%%%%%%%%%%%%%%%%%%%%%%%%%%%%%%%%%

\subsubsection{Final Regression Model by Age Category for MUPO Subset}

Separate regressions were conducted for each age category using the set of
predictors from the final model. Table 5 provides the OLS parameter estimates 
with standardized betas in parentheses, F-test, $R^2$, adjusted $R^2$, and 
significance level of the estimates. As shown in Table 5, the regression 
model F-test was significant for every age group; however, the significance
of the parameter estimates for the regressors varied by age group. The 
standardized betas are given in parentheses as an indication of the relative 
contribution of each predictor. 

In general, the influence of demographic features and illicit drugs varied 
considerably across age groups; In terms of demographic factors, Sex was a 
significant predictor of PRLMISAB between 18 to 36 years and over age 50, with
the largest effect observed between 26 to 35 years. Marital status 
significantly predicted PRLMISAB only between the ages of 18-26, whereas 
education level significantly predicted PRLMISAB only for respondents age 
36 years and older, with the largest effect at age 50 and older. As for 
illicit drugs, Heroin Use significantly predicted PRLMISAB between the ages 
of 18 to 49 years; with a larger effect between 26 to 49 years. Amphetamines 
significantly predicted PRLMISAB between 12 to 35 years, with the largest 
effect observed in the youngest age group. Surprisingly, Cocaine was not
a significant predictor of PRLMISAB at any age group, although it was a 
significant predictor when age was included as a regressor in the final model. 

Of the medications and illicit drugs included in the final model with the
MUPO subset, the use of any opioid pain relievers was the only predictor of
PRLMISAB that was significant predicted  across all age groups, and 
contributed the greatest amount in the change/effect in the dependent variable 
than any other regressor for the ages of 18 and above. For the youngest age
group, between 12 to 17 years, Amphetamine use contributed more to change in 
PRLMISAB than Use of Any opioid Pain Relievers. Between 18 to 26 years, use of
any opioid pain relievers contributed more than Marital Status, Sex, 
Tranquilizers, Amphetamines, or Heroin (in descending order). Between 26 to 35
years, any opioid pain reliever contributed more than Sex which contributed the 
second largest portion of change in the outcome, followed by Heroin, 
Tranquilizers, and Amphetamines. Between 36 to 49 years, use of any opioid pain 
relievers accounted for more change in the dependent variable than Heroin, 
Tranquilizers, and Education level. Finally, for respondents age 50 and up, 
any opioid pain reliever use conributed more than Education level, which 
accounted for the second largest amount of change in PRLMISAB, followed by 
Tranquilizers, and Sex, in decreasing order of contribution. 


%%%%%%%%%%%%%%%%%%%%%%%%%%%%%%%%%%%%%%%%%%%%%%%%%%%%%%%%%%%%%%%%%%%%%%%%%%%%%%%%

\section{Discussion}

The results indicated that approximately one-quarter of individuals who had 
taken any prescribed opioid pain relievers reported misusing them, which 
replicates past findings. Pain reliever misuse and abuse varied by age, with 
the highest rates of misuse between 18 to 25 years, and decreased among older 
age groups. Demographic factors such as age, sex, marital status, and education 
level were associated with decreased pain reliever misuse and abuse, and the 
contribution of these demographic factors varied across age groups: Sex 
predicted PRLMISAB between 18 to 35 years, with the largest effect between the 
ages of 26 to 35. Marital status predicted PRLMISAB only between 18 to 25 years. 
Education level predicted PRLMISAB from age 36 and up, but had the largest 
contribution for the ages of 50 and older. As predicted, the use of medications 
and illicit drugs was related to increased misuse of pain relievers. Use of any 
opioid pain relievers predicted PRLMISAB across all ages, followed by Tranqulizer 
use, which predicted PRLMISAB for ages 18 and older. As predicted, misuse and 
abuse of pain relievers was associated with heroin use, as the proportion of 
pain reliever misuse and abuse was higher for individuals who had used heroin 
than those who had not. However, in the final model, with the MUPO subset, 
cocaine and amphetamines were more influential predictors of PRLMISAB than heroin.
The effect of illicit drugs varied by age. Heroin predictor of PRLMISAB between 
the ages 18 to 36 years; however, Amphetamines contributed more to predicting 
PRLMISAB between 12 to 17 years, than any opioid pain reliever misuse. Although 
Cocaine predicted PRLMISAB when age was included in the final model, it was 
not a significant predictor for any of the regression models considered by age 
group separately. It is interesting to note that, in the final model, health, 
mental health, treatment, and counseling did not predict PRLMISAB, although 
these regressors were significant predictors in the initial regression 
model with the full sample. 

%%%%%%%%%%%%%%%%%%%%%%%%%%%%%%%%%%%%%%%%%%%%%%%%%%%%%%%%%%%%%%%%%%%%%%%%%%%%%%%%
\subsection{Limitations}

In selecting the best set of variables to model pain reliever misuse and abuse, 
the resulting tradeoff was a decrease in the proportion of variance accounted
for ($R^2$) due to a reduction in sample size and number of independent 
variables in each successive model. The final regression model provided a 
parsimonious and interpretable model of opioid misuse and abuse, though it 
was based on a smaller sample, with a lower coefficient of determination. A 
limitation of the study is that misspecification may occur due to the exclusion 
of relevant variables such as socioeconomic status, ethnicity, or region
from the model. In future, related proxy variables could be included to 
represent variance in the relationship of those missing variables with the 
dependent measure. Using aggregated variables may have masked the separate 
contributions of individual features. A drawback of assessing pain reliever 
misuse and abuse on a scale, taken as the sum of several binary responses, 
is that many aggregated variables had bimodal distributions that were highly 
skewed.  It may have been appropriate to analyze the data using a logit model 
to obtain the log odds ratios of opioid pain reliever misuse and abuse. An 
initial goal of the study was to analyze trends in prescription opioid misuse 
over time using panel data. The sample used in this study included NSDUH for 
only two years, which may have limited the generalizability of the findings. 
In future, it would be preferable to combine NSDUH data across multiple years 
to construct a more complete data set, with an inclusive set of variables, 
to build more informative models of opioid misuse and abuse.

%%%%%%%%%%%%%%%%%%%%%%%%%%%%%%%%%%%%%%%%%%%%%%%%%%%%%%%%%%%%%%%%%%%%%%%%%%%%%%%%
\section{Conclusion}

In conclusion, demographic factors, medications, and illicit drugs contributed 
differently to the aggregated measure of prescription opioid misuse and abuse. 
The findings are consistent with the idea that personal characteristics may 
play a role in helping to prevent or decrease the misuse and abuse of opioids
pain relievers. Overall, respondents who were older, female, married, with more
education, were less likely to misuse and abuse opioid pain relievers than 
their counterparts. By contrast, any previous prescription opioid use, 
tranquilizer use, heroin, cocaine, or amphetamine use put people at increased 
risk for opioid misuse and abuse. The widespread availability and prescription 
of opioids for chronic pain is likely a primary contributing factor driving the 
rapid escalation of opioid addiction and abuse. On the other hand, the misuse 
and abuse of opioids may be influenced by risk factors that contribute to 
psychological and physical dependence. Theories of addiction suggest that 
situational cues or events can elicit a motivational state underlying relapse, 
as addictive behavior are reinstated by exposure to drugs, drug-related cues, or 
environmental stressors \cite{shaham03}. A lack of continuity in treatment also 
leaves many people in recovery at risk for relapse or possible overdose as they 
return to the same environments and social relationships associated with their 
drug use. The sharp increase in overdose deaths in the U.S. due to synthetic 
opioids (other than methadone) has coincided with the increased availability 
of illicit synthetic opioid, fentanyl \cite{nida17}. Because the dosage levels 
and potency of illicit opioids are largely unknown, there is greater risk of 
drug overdose and death. This study may help raise awareness about personal 
characteristics that help individuals resist opioid misuse and dependency. 
Additional research is needed to understand factors related to resilience
in overcoming opioid abuse and addiction. 


%%%%%%%%%%%%%%%%%%%%%%%%%%%%%%%%%%%%%%%%%%%%%%%%%%%%%%%%%%%%%%%%%%%%%%%%%%%%%%%%

\begin{acks}

Portions of this paper were completed as part of a course project in Data 
Analysis and Modeling taught by Dr. Barry Rubin (SPEA Connect P507) at 
Indiana University in Spring 2018. The author would like to thank Professor 
Rubin and Dr. Venkat Nadella, the TA, for helpful guidance and feedback on 
the project proposal and analysis. Thanks to Taylor Roundtree for conducting
tests and corrections for heteroscedasticity.

\end{acks}
%ACM-Reference-Format
\bibliographystyle{ACM-Reference-Format}
\bibliography{report} 

%%%%%%%%%%%%%%%%%%%%%%%%%%%%%%%%%%%%%%%%%%%%%%%%%%%%%%%%%%%%%%%%%%%%%%%%%%%%%%%%

\appendix

\section{Technical Appendix}

Note: Tests of the OLS assumptions reported below were conducted using 
a data subset with additional outliers removed, and therefore the sample 
\textit{N}'s reported in this section are different from the sample sizes 
reported in the body of the paper.

\subsection{Detection and Correction of Heteroscedasticity}

There was some concern for heteroscedasticity in the data because of the 
demographic variables measured in cross-section, high degree of skew in the 
aggregated variables, presence of outliers, and non-normal distributions of 
errors. An auxiliary regression conducted for the White`s test yielded a 
$R^2$ = 0.096, with N = 24,963 and 121 parameters, which provided a $\chi^2$
= 2401.44, that was greater than the critical value of $\chi^2$ = 124.34 
(df=100), at the $\alpha$ = 0.05; the null hypothesis was rejected, 
indicating evidence of strong heteroscedasticity in the data.  

The Goldfield-Quandt Test was conducted using \textit{C} = 2952 (16 parameters), 
which resulted in a significant F-test result of 1.24 which was greater than 
the critical value of \textit{F}(500,200) = 1.22, at the $\alpha$ = 0.05 level; 
the null hypothesis was rejected providing evidence of heteroscedasticity. 
To correct for heteroscedasticity, the Weighted Least Squares (WLS) 
transformation was used based on the assumption that $\alpha^2$ = 
$\alpha^2((AGECAT)^2)$, with PROC MODEL in SAS to obtain the WLS regression 
equation and White`s test for this new model. The White`s test from PROC 
MODEL WLS regression corrected for heteroscedasticity yielded a $\chi^2$ = 2081, 
\textit{p} $<$ 0.0001, which indicated that heteroscedasticity was reduced but 
not eliminated from the data.

The same procedure was conducted with the MUPO subset. The auxiliary 
regression for the White`s test yielded an $R^2$ = 0.064 (\textit{N} = 6478, 
121 parameters), which provided a $\chi^2$ = 414.59, that was greater than 
the critical value of $\chi^2$ = 124.34 (df=100), at the $\alpha$ = 0.05 level; 
the null hypothesis was rejected, indicating evidence of heteroscedasticity.  
The Goldfield-Quandt Test was conducted using \textit{C} = 1505 (16 parameters), 
which resulted in a \textit{F} = 1.10, which was less than the critical value 
of \textit{F}(500,200)= 1.22, at the $\alpha$ = 0.05 level; this non-significant 
result suggested there was not sufficient evidence of heteroscedasticity. 
However, the WLS transformations was used based on the assumption that 
Age Category was contributing to heteroscedasticity in the data.


\end{document}
