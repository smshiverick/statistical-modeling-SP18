\documentclass[sigconf]{acmart}

\input{format/final}

\begin{document}
\title{Modeling Opioid Pain Reliever Misuse and Abuse}
  
  \author{Sean M. Shiverick}
  \affiliation{
  \institution{Indiana University Bloomington}
  }
\email{smshiver@iu.edu}

\renewcommand{\shortauthors}{S.M. Shiverick}

%%%%%%%%%%%%%%%%%%%%%%%%%%%%%%%%%%%%%%%%%%%%%%%%%%%%%%%%%%%%%%%%%%%%%%%%%%%%%%%%

\begin{abstract}
WORKING DRAFT - UNDER REVISION

The misuse and abuse of prescription opioids is influenced by both personal 
characteristics and environmental factors. In this study, data from the 
National Survey on Drug Use and Health (NSDUH 2015-16) was used to regress 
pain reliever use, misuse, abuse, and dependency on demographic features and
aggregated measures of medication use, illicit drugs, treatment, and counseling.  
In a subset of N=6946 individuals who reported misuse and abuse of prescription 
opioids, age group, sex, marital status, and education were negatively related
to pain reliever misuse and abuse. By contrast, Use of opioid pain relievers and 
tranquilizers was positively related to pain reliever misuse and abuse across 
all age groups. Use of cocaine and amphetamines was more influential than 
heroin use in predicting pain reliever misuse and abuse; however, the effects 
of illicit drugs varied by age group. Heroin use was significantly correlated
with pain reliever misuse and abuse between the ages of 18 and 49 years. Use of 
amphetamines was significantly related to pain reliever misuse and abuse 
between 12 and 35 years of age. These findings suggesting that some demographic 
characteristics may be preventative of opioid misuse, and may contributed to 
personal resilience in overcoming opioid addiction.
\end{abstract}

\keywords{Opioid Misuse, Pain Reliever Abuse, Linear Regression Models}

\maketitle

%%%%%%%%%%%%%%%%%%%%%%%%%%%%%%%%%%%%%%%%%%%%%%%%%%%%%%%%%%%%%%%%%%%%%%%%%%%%%%%%
\section{Introduction}

Over the past two decades, prescription opioid misuse, abuse, and addiction 
in the U.S. has become a major crisis with serious public health consequences.
In 2015, an estimated 2 million Americans suffered from substance use disorders 
related to opioid pain medications such as oxycodone and hydrocodone 
\cite{nida18,cdc18}. Of patients legitimately prescribed opioids for chronic 
pain, approximately 25\% misused them, between 8\% to 12\% became addicted, and 
4\% to 6\% transitioned to heroin \cite{vowles15, carlson16}. Opioid dependence 
and addiction are chronic health conditions; following treatment, many addicted 
individuals are at high risk for relapse and overdose death \cite{shaham03}. 
Since 1999, the number of overdose deaths from prescription opioids has more 
than quadrupled \cite{cdc16}. The economic cost of the opioid crisis since 2001 
is estimated to exceed one trillion dollars \cite{altarum18}. 

Past studies have identified risk factors for opioid abuse, including gender, 
ethnicity, comorbid psychological disorders, and non-opioid drug abuse 
\cite{yokell13,rice12}. Rates of hospitalization for prescription opiate overdose 
(POD) are higher for females than males, and the increase in POD is highest for
Whites compared to Blacks or Hispanics \cite{unick13}. Supply-based interventions 
to reduce the availability of prescription opioids have produced a shift to 
heroin use, and the exponential increase in POD and heroin overdose deaths (HOD) 
are correlated \cite{jones15,reifler12}. The non-medical use of prescription 
opioids may also vary by age \cite{mccabe12}. This study modeled the 
relationships between demographic features, medications, and illicit drugs as 
predictors of pain reliever misuse and abuse. Identifying factors which 
are negatively related with pain reliever misuse may reveal preventative 
characteristics that decrease opioid dependence and addiction. Models of pain 
reliever misuse and abuse are compared across age groups. 

%%%%%%%%%%%%%%%%%%%%%%%%%%%%%%%%%%%%%%%%%%%%%%%%%%%%%%%%%%%%%%%%%%%%%%%%%%%%%%%%

Many researchers have analyzed the probability that an individual abused 
or misused opioids (or did not) with logistic regression 
\cite{rice12, unick13, jones15, mccabe12}. The logistic model takes the form:
$ Pr(abuse=1) = F(\alpha+\sum_j(\beta_i*X_ij)) $, where the dependent variable
is a binary outcome. In the present study, pain reliever misuse and abuse was 
assessed along a continuum by aggregating responses for several related binary 
responses into a single measure. Similarly, responses for related independent 
measures were summed, creating several aggregated variables as regressors in a 
linear regression model rather than a logit model. The general linear model 
(Ordinary Least Squares) estimates the coefficients of the independent 
variables in relation the the dependent variable, with some degree of error. 

\begin{equation}
  \ Y_i = \beta_1*X_1 + \beta_2*X_2 +... + \beta_k*X_k + u_i
\end{equation}

\subsection{Generalized Least Squares Model}

The classical linear regression model is based on several assumptions, 
including: the model is linear in the parameters, the fixed x-values are 
independent of the error term, the mean value of the error terms is zero, 
non-collinearity among the independent variables, and independent 
observations (i.e., non-autocorrelation). An important assumption for the 
present study, is that the disturbances $u_i$ all have the same variances 
(i.e., \emph{homoscedasticity}) \citeN{gujarati09}. Heteroscedastic 
(\emph{non-equal}) variances is a common occurrence in socio-economic 
variables measured in cross-section, but can also arise due to skewness in 
the distribution of one or more regressors in the model, or from incorrect 
functional form (e.g., linear versus log-linear models). Outliers can also 
contribute to heteroscedasticity. If the assumption of homoscedasticity is 
not met, the ordinary least squares (OLS) parameter estimates (though still 
linear and unbiased) are no longer ``best'', are not efficient as they do 
not provide the minimum variance, and yield unreliable statistics. 

In the presence of heteroscedasticity, Weighted Least Squares (WLS)-a subset 
of Generalized Least Squares (GLS)-provides estimates with a smaller variance 
to replace the OLS estimators.  To control for heteroscedasticity using WLS, 
first the primary contributing variable(s) must be identified; then an 
appropriate transformation is carried out on the data to correct for 
heteroscedasicity. OLS is then applied to the transformed data. WLS provides 
estimates which are `best', `linear', `unbiased' (i.e., BLUE). A common 
ad hoc assumption with WLS is that the error variance is proportional to 
the square of one of the explanatory variables $\sigma^2=\sigma^2 X_ji^2$ 
\cite{gujarati09}. The originaal model is transformed by dividing through 
by the explanatory variable $X_i$:

\begin{equation}
  \ Y_i/X_2 = \beta_1*X_2 + \beta_2 +... + \beta_k*X_k/X_2 + u_i/X_2
\end{equation}

%%%%%%%%%%%%%%%%%%%%%%%%%%%%%%%%%%%%%%%%%%%%%%%%%%%%%%%%%%%%%%%%%%%%%%%%%%%%%%%%

\subsection{National Survey on Drug Use and Health} 

Data for the study was derived from the National Survey on Drug Use and Health 
(NSDUH) from 2015 and 2016, obtained from the Substance Abuse and Mental Health 
Data Archive (SAMHDA) \cite{samhsa16}. The NSDUH is a large survey that consists
of approximately 2600 questions covering all aspects of substance use, misuse, 
dependency, abuse, and addiction for prescription medications and illicit drugs, 
with a comprehensive set of demographic questions related to physical health, 
depression, mental health treatment, and drug and alcohol treatment. According 
to the NSDUH codebook, sampling was weighted across states by population size 
for a representative distribution selected from 6,000 area segments. The sample 
design used five state sample size groups, drawing more heavily from the eight 
states with the largest population (e.g., CA, FL, IL, MI, NY, OH, PA, TX) which 
together account for 48 percent of the total U.S. population aged 12 or older.  
All identifying information in the public data set were collapsed (e.g., age 
categories) and state identifiers were removed from the public use file to 
ensure confidentiality. The NSDUH public-use files do not include geographic 
location, or demographic variables related to ethnicity or immigration status. 
The weighted survey screening response rate was 81.94 percent and the weighted 
interview response rate was 71.2 percent. For the present study, approximately
90 variables were selected that were aggregated to a data set of 20 variables. 


%%%%%%%%%%%%%%%%%%%%%%%%%%%%%%%%%%%%%%%%%%%%%%%%%%%%%%%%%%%%%%%%%%%%%%%%%%%%%%%%
\subsection{Study Goals} 

The main goal was to assess demographic and environmental features as 
predictors of pain reliever misuse and abuse. It was hypothesized that 
demographic variables such as marital status and education level may play a 
preventive role associated with decreased pain reliever misuse and abuse. 
Based on past findings that misuse of prescription opioids varies by age, 
age group was an important covariant in the model, which was believed to be a
primary contributor to heteroscedasticity; WLS transformations were used to 
correct for heteroscedasticity due to age category. Pain reliever misuse and 
abuse was expected to decrease with age. Use of non-opiod medications and illicit 
drugs was predicted to be positively associated with pain reliever misuse; 
heroinuse was expected to be highly correlated with pain reliever misuse and abuse. 
The analysis also examined the subset of individuals who reported ever misusing 
or abusing pain relievers in the past. Finally, separate regression models were 
constructed for each age group to examine differences in the relative influence 
of predictors across different age subsets. 

%%%%%%%%%%%%%%%%%%%%%%%%%%%%%%%%%%%%%%%%%%%%%%%%%%%%%%%%%%%%%%%%%%%%%%%%%%%%%%%%
\section{Method}

The project workflow pipeline included the following steps: (1) Data cleaning and 
preparation, (3) Exploratory data analysis, (3) Data Visualization, and (4) 
Construction of regression models for opioid pain reliever misuse and abuse. 

\subsection{Data Cleaning and Preparation }

The data was extracted
and prepared using an interactive Python notebook based on examples in 
\emph{Python for Data Analysis} \cite{mckinney17}. NSDUH data files for 2015 
and 2016 were downloaded from the SAMHDA website \cite{samhsa16} URL, and 
saved as data frame objects in Pandas. The initial data files consisted of 
57,146 observations from 2015, and 56,897 observations for 2016, with over 
2,600 features. The data was subset by columns, selecting approximately 90 
variables that included demographic information, health, mental health, 
medication use, illicit drug use, and treatment for drugs or alcohol or 
mental health. The following steps were taken to detect and remove 
inconsistencies in the data: (a) Remove missing values (i.e., `NaN'); (b) 
Recode blanks, non-responses, or legitimate skips (e.g., 99, 991, 993) to 
zero; (c) Recode dichotomous responses (Yes=1 / No=2) so that No=0; (d) 
Recode categorical variables to be consistent with amount or degree 
(1=low, 2=med, 3=high); (e) Rename selected variables for better 
description (e.g., Major Depressive Episode Lifetime = `DEPMELT').

\subsubsection{Aggregated Variables} 

Responses for related binomial variables (i.e., No=0 / Yes=1) were summed to 
create several aggregated variables; Table 1 lists the initial set of
variables and range of scores. For example, a measure of overall health 
was created by combining any previous health problems (e.g., STDs, Hepatitis,
HIV, Cancer, Hospitalization). Overall mental health was assessed as a   
combination of any adult major depressive episode, severe emotional distress, 
suicidal thoughts, or plans. The use of any prescription opioid pain reliever 
medications taken in past year (`PRLANY`) was assessed by summing ten of the 
most commonly used opioids (e.g., Hydrocodone, Oxycodone, Tramadol, Morphine, 
Fentanyl, Oxymorphone, Demerol, Hydromorphone). The main dependent variable, 
pain reliever misuse and abuse (PRLMISAB), was based on the self-reported 
non-medical use of prescription opioids not directed by a physician, use of 
pain relievers in amounts greater than prescribed, misuse of pain relievers in 
the past year or month, dependence or abuse of pain relievers, and use of pain 
relievers to ``get high''. Similar aggregated measures were constructed for the 
use and misuse of prescription Tranquilizers, Sedatives, Heroin (HEROINUSE), 
Cocaine, Amphetamines, and Hallucinogens. Drug or alcohol treatment was summed 
across different treatment contexts (e.g., Inpatient, Outpatient, Hospital, 
MH Clinic, ER, Drug Treatment Status). A data codebook describes the list of 
variables summed in the aggregated variables \citeN{shiverick18}. 

\begin{table*}[ht]
  \caption{Summary of Variables in Initial Data Set (16 Independent Variables)}
  \label{tab:freq}
  \begin{tabular}{ll}
    \toprule
    \textit{Dependent Variable} & Label \\
    \midrule
    Prescription Opioid Pain Reliever Misuse and Abuse (0-12 Scale)& PRLMISAB  \\
    \midrule
    \textit{Demographic Variables}&   \\
    \midrule
    Age Category (1=12-17 years, 2=18-25, 3=26-34, 4=35-49, 5=50 and older)& AGECAT \\
    Biological Sex (0=Male, 1=Female)& SEX  \\
    Marital Status (0=Unmarried, 1=Divorced, 2=Widowed, 3=Married)& MARRIED  \\
    Education (1=Less H.S., 2=H.S. Grad., 3=Some College,  4=College Grad.)& EDUCAT  \\
    Size of City/Metropolitan Region (1=Rural, 2=Small, 3=Large)& CTYMETRO  \\
    Health Problems Aggregated  (0-10 scale)& HEALTH  \\
    Mental Health, Aggregated: adult depression, emotional distress (0-10 scale)& MENTHLTH  \\
    Treatment for Drugs and Alcohol in past year, Aggregated (0-5 scale)& TRTMENT  \\
    Mental Health Treatment, Aggregated (Likert scale, 1-10)& MHTRTMT  \\
    \midrule
    \textit{Medication and Drug Use Variables}& \\
    \midrule
    Any Prescribed Opioid Pain Reliever Use in past year, Aggregated (1-10 scale)& PRLANY  \\
    Tranquilizer use, past year, Aggregated (Likert scale, 0-5)& TRQLZRS \\
    Sedative use, past year, Aggregated (0-5 scale)& SEDATVS  \\
    Heroin use, past year, Aggregated (0-5 scale)& HEROINUSE  \\
    Cocaine and Crack Cocaine Use in past year, Aggregated  (0-5 scale)& COCAINE  \\
    Amphetamine and Methamphetamine Use in past year, Aggregated (0-5 scale)& AMPHETMN  \\
    Hallucinogen Use in past year, Aggregated (0-5 scale)& HALUCNG \\
    \bottomrule
  \end{tabular}
\end{table*}

%%%%%%%%%%%%%%%%%%%%%%%%%%%%%%%%%%%%%%%%%%%%%%%%%%%%%%%%%%%%%%%%%%%%%%%%%%%%%%%%%

\subsubsection{Data Subsets} 
 
The data files were subset to select only respondents who indicated using
any prescription opioid pain reliever; 13,916 individuals (24.35\%) from 2015 
and 11,690 individuals (20.54\%) from 2016 were combined, resulting in an
initial sample of N=25,606 observations, as a pooled, cross-sectional data set.
Preliminary examination of the data revealed 174 outliers that exceeded the 
range of the dependent variable and were excluded. The revised sample consisting
of N=25,432 observations (10655 male, 14777 female) with 17 variables was 
exported to CSV file entitled `project-data.csv`. An additional data subset
consisted of N=6,964 individuals who reported misusing or abusing prescription 
opioids (i.e.,``MUPO'') at some point. 

%%%%%%%%%%%%%%%%%%%%%%%%%%%%%%%%%%%%%%%%%%%%%%%%%%%%%%%%%%%%%%%%%%%%%%%%%%%%%%%%
\section{Results}

\subsection{Exploratory Data Analysis}

Overall, the majority of individuals reported never misusing or abusing 
prescription opioids, but 27.25\% of individuals reported pain reliever misuse
and abuse. Figure 1 shows that the proportion of pain reliever misuse 
or abuse was highest for individuals between 18 to 25 years, and decreased
with age. In general, pain reliever misuse and abuse was higher among younger 
than older age groups. More women (61\%) than men (39\%) reported using any
prescription opioid pain relievers, but equal proportions of men and women 
(50\%) reported misuse and abuse of opioid pain relievers. Table 2 provides 
the summary statistics of aggregated variables measured on a scale for both the 
full sample and the MUPO subset. The mean and standard deviation for the 
dependent variable are more robust in the MUPO subset than the full sample.  

\begin{figure}[!ht]
  \centering\includegraphics[width=\columnwidth]{images/Figure1.pdf}
  \caption{Proportion of Individuals Reporting Opioid Pain Reliever Misuse 
  or Abuse (PRLMISAB) by Age Group}
  \label{f:Figure1}
\end{figure}

Figure 2 shows the proportion of individuals who reported misusing prescription 
opioid pain relievers and also using heroin. The left column of the Figure 1 
shows the majority of respondents (89 percent) stated they had never misused 
opioid pain medication or used heroin, although 10 percent reported misusing 
opioid pain medication at some point. The right panel of Figure 1 shows that, 
of those individuals who reported using heroin, the proportion who reported 
misusing opioid pain medication was roughly twice as large as the proportion 
who reported only using heroin. This is consistent with the hypothesized 
link misuse of prescription opioids and heroin use. 

\begin{table*}[ht]
  \caption{Summary Statistics for Aggregated Variables for Full Sample and 
  Subset of Individuals Reporting Misuse and Abuse of Prescription Opioids (MUPO)}
  \label{tab:freq}
  \begin{tabular}{llllllll}
    \toprule
     & Full Sample& (N = 25,432)&&& MUPO Subset& (N = 6,946)&  \\
    \midrule
    Variables & Mean& S.D.& Median& Range& Mean& S.D.& Median  \\
    \midrule
    PRL Misuse Abuse& 1.74& 3.30& 0& 12& 6.38& 3.11 & 7 \\
    Health Problems& 2.77& 1.15& 3& 7& 2.71& 1.11& 3  \\
    Mental Health& 1.57& 2.37& 0& 12& 2.21& 2.71& 1  \\
    Drug Treatment& 0.15& 0.89& 0& 10& 0.43& 1.48& 0  \\
    MH Treatment& 0.37& 0.86& 0& 7& 0.48& 0.97& 0  \\
    Any PRL Use& 1.61& 1.06& 1& 10& 1.93& 1.37& 1 \\
    Tranquilizer Use& 0.60& 1.22& 0& 5& 1.05& 1.42& 0  \\
    Sedative Use& 0.12& 0.50& 0& 5& 0.18& 0.62& 0  \\
    Heroin Use& 0.10& 0.49& 0& 5& 0.30& 0.85& 0  \\
    Cocaine use& 0.38& 0.80& 0& 5& 0.83& 1.10& 0 \\
    Amphetamines& 0.32& 0.72& 0& 5& 0.73& 1.04& 0  \\
    Hallucinogens& 0.65& 1.16& 0& 5& 1.42& 1.47& 1  \\
    \bottomrule
  \end{tabular}
\end{table*}

%%%%%%%%%%%%%%%%%%%%%%%%%%%%%%%%%%%%%%%%%%%%%%%%%%%%%%%%%%%%%%%%%%%%%%%%%%%%%%%%

Although relatively few individuals in the sample reported ever using heroin 
(4.7\% overall, 692 males, 505 females), Figure 2 shows that the proportion of 
pain reliever misuse and abuse was higher for individuals who had used heroin 
than for those who had not, which is consistent with the hypothesized link 
between misuse of prescription opioids and heroin use. (Note: some predictor 
variables used for EDA were excluded from regression analysis;
e.g, PRLMISEVR, HEROINEVR). 

\begin{figure}[!ht]
  \centering\includegraphics[width=\columnwidth]{images/Figure2.pdf}
  \caption{Proportion of Individuals Reporting Pain Reliever Misuse
  and Abuse and Heroin Use (N=25,432}
  \label{f:Figure2}
\end{figure}


%%%%%%%%%%%%%%%%%%%%%%%%%%%%%%%%%%%%%%%%%%%%%%%%%%%%%%%%%%%%%%%%%%%%%%%%%%%%%%%%

\subsection{Linear Regression Models}

\subsubsection{Initial OLS Regression on the Full Sample} 
 
After loading the project data into SAS, pain reliever misuse and abuse was 
regressed on the 16 independent variables using OLS with the full sample; 
this regression was statistically significant, \textit{F}(16, 24947) = 
520.47, \textit{p} $<$ 0.0001, ($R^2$ = 0.2504, adjusted $R^2$ = 0.2498). The 
variance inflation values and condition index were all within acceptable ranges, 
and the t-values for all but one independent variable in the OLS model was 
statistically significant, which indicated that near multicollinearity was not 
an issue in the data set. Examination of the residuals plotted against the 
predicted values of the dependent variable revealed an additional outlier 
that was removed. The White’s test and Goldfield-Quandt test revealed 
heteroscedasticity in the data. The Durbin-Watson test was statistically 
significant, which indicated autocorrelation in the model, likely due to 
model misspecification. Heteroscedasticity was addressed using Weighted 
Least Squares (WLS), based on the assumption that Age Category was primarily 
responsible for heteroscedasticity in the data. 

\subsubsection{Weighted Least Squares} 

The WLS regression was conducted using PROC MODEL in SAS, yielded a 
statistically significant relationship between Pain Reliever Misuse and Abuse 
and the independent variables, \textit{F}(16, 24947) = 542, \textit{p} $<$ 
0.0001 ($R^2$ = 0.2437, adjusted $R^2$ = 0.2425). The WLS transformation 
reduced heteroscedasticity from the initial OLS model, but it was not 
completely eliminated from the data. The WLS parameter estimates for the full 
sample given on the left side of Table 3 show that Education Level, Size of 
City/Metropolitan area, and Health Problems  were not significant predictors of
pain reliever misuse and abuse. To take a more focused look at opioid abuse,
the next analysis focused on the subset of misuse and abuse of 
prescription opioids (MUPO). 

%%%%%%%%%%%%%%%%%%%%%%%%%%%%%%%%%%%%%%%%%%%%%%%%%%%%%%%%%%%%%%%%%%%%%%%%%%%%%%%%

\subsubsection{WLS Regression for the MUPO Subset} 

The following section reports the regression analyses for the subset of 
N=6,478 individuals who reported MUPO. The OLS regression on the MUPO subset 
was statistically significant, \textit{F}(16, 6461) = 51.69, \textit{p} 
$<$ 0.0001, ($R^2$ = 0.1135, adjusted $R^2$ = 0.1113). The White’s test and 
Goldfield-Quandt tests were both statistically significant, indicating that
the heteroscedasticity was an issue in the MUPO subset, and the OLS estimates 
would not be best or efficient. The WLS transformation was used to correct 
for heteroscedasticity due to Age Group. The WLS regression revealed a 
statistically significant relationship between Pain Reliever Misuse and 
Abuse and the independent variables, \textit{F}(16, 6462) = 51.95, 
\textit{p} < 0.0001 ($R^2$ = 0.1076, adjusted $R^2$ = 0.1056). WLS parameter 
for the MUPO subset are given in the right side of Table 3, which shows that
the key demographic features were negatively related to pain reliever misuse 
and abuse, whereas medication use and use of illicit drugs was positively 
related to pain reliever misuse and abuse. 

\begin{table*}[ht]
  \caption{Weighted Least Squares (WLS) Parameter Estimates for Regression 
  of Pain Reliever Misuse and Abuse}
  \label{tab:freq}
  \begin{tabular}{lllllll}
    \toprule
     & Full Sample& (N = 25,432)&& MUPO Subset& (N = 6,946)&  \\
    \midrule
    Variables & Estimate& S.E.& t-value& Estimate& S.E.& t-Value  \\
    \midrule
    Intercept& 1.656***& 0.083& 19.85& 6.870***& 0.154& 44.72 \\
    Age Category& \textbf{-0.370}& 0.028& -13.23& \textbf{-0.312}***& 0.059& -5.31 \\
    Sex& \textbf{-0.156}***& 0.039& -4.03& \textbf{-0.154}***& 0.069& -2.24  \\
    Married& \textbf{-0.041}**& 0.021& -2.02& \textbf{-0.159}***& 0.039& -4.03  \\
    Education& -0.028& 0.020& -1.41& \textbf{-0.123}***& 0.037& -3.36 \\
    City/Metro Size& -0.016& 0.025& -0.67& 0.017& 0.043& 0.40 \\
    Health Problems& -0.021& 0.019& -1.12& -0.041& 0.034& -1.22 \\
    Mental Health& \textbf{0.059}***& 0.011& 5.22& -0.022& 0.018& -1.25 \\
    Drug Treatment& \textbf{0.190}***& 0.025& 7.61& -0.13& 0.024& -0.57 \\
    MH Treatment& \textbf{-0.143}***& 0.033& -4.39& 0.017& 0.029& 0.33 \\
    Any PRL Use& \textbf{0.252}***& 0.020& 12.62& \textbf{0.290}***& 0.029& 9.87 \\
    Tranquilizer Use& \textbf{0.447}***& 0.021& 21.40& \textbf{0.116}***& 0.029& 4.05 \\
    Sedative Use& \textbf{0.192}***& 0.039& 4.95& 0.043& 0.053& 0.81 \\
    Heroin Use& \textbf{0.358}***& 0.055& 6.52& \textbf{0.258}***& 0.059& 4.36 \\
    Cocaine Use& \textbf{0.350}***& 0.035& 9.95& \textbf{0.103}***& 0.043& 2.38 \\
    Amphetamines& \textbf{0.780}***& 0.030& 25.93& \textbf{0.232}***& 0.037& 6.23 \\
    Hallucinogens& \textbf{0.576}***& 0.023& 25.66& 0.006& 0.030& 0.21 \\
    \bottomrule
    Significance level:& **0.05,& ***0.01&&&&
  \end{tabular}
\end{table*}

%%%%%%%%%%%%%%%%%%%%%%%%%%%%%%%%%%%%%%%%%%%%%%%%%%%%%%%%%%%%%%%%%%%%%%%%%%%%%%%%

\subsubsection{Final WLS Model for the MUPO subset}

The final WLS model was obtained by excluding the non-significant independent 
variables from the previous WLS regression on the MUPO subset. The final WLS 
regression demonstrated a significant relationship between Pain Reliever Misuse 
and Abuse and the independent variables, \textit{F}(16, 6462) = 51.95, 
\textit{p} $<$ 0.0001 ($R^2$ = 0.1071, adjusted $R^2$ = 0.0865). Overall, 
10.71\% of the variation in the predicted value of Pain Reliever Misuse and 
Abuse in the MUPO subset was due to changes in the independent variables; 
and 8.65\% of the variability in Pain Reliever Misuse and Abuse was accounted 
for, taking into account the number of independent variables. The parameter 
estimates for the final WLS model are given in Table 4. 

As seen in Table 4, demographic variables were negatively correlated with Pain 
Reliever Misuse and Abuse (PRLMISAM): A one-unit change in Age Category yielded a 
-0.327 decrease in the predicted value of PRLMISAB, holding the effect of other 
independent variables constant. Similarly, there was a -0.17 decrease in PRLMISAB 
for females compared to males, all other factors considered equal. A one-unit 
change in Marital Status was associated with -0.158 decrease in PRLMA, holding
constant other variables. And a one-unit increase in Educational Level was 
associated with -0.130 decrease in PRLMA, with other predictors held constant. 

Use of prescription medications and illicit drugs was positively correlated with 
Pain Reliever Misuse and Abuse in the MUPO subset. A one-unit increase in Any 
Pain Reliever Misuse or Abuse yielded a 0.289 unit increase in PRLMA, holding 
other variables constant. Likewise, a one-unit increase in Tranquilizer use was 
linked to a 0.114 unit increase in PRLMA. In terms of illicit drugs, a one-unit 
increase in Heroin use was associated with a 0.244 unit increase in PRLMA as 
expected, holding constant the effects of other independent variables. 
A one-unit increase in Cocaine use was associated with a 0.102 increase in PRLMA. 
Finally, a one-unit increase in Amphetamine use was related to a 0.231 unit 
increase in Pain Reliever Misuse and Abuse.  

\begin{table}
  \caption{WLS Parameter Estimates for Final Model of Pain Reliever Misuse and Abuse 
  with MUPO Subset (N=6964)}
  \label{tab:freq}
  \begin{tabular}{llll}
    \toprule
    Variable & Estimate& S.E.& t-Statistic \\
    \midrule
    Intercept& 6.850***& 0.098& 69.69 \\
    Age Category& \textbf{-0.327}***& 0.058& -5.64 \\
    Sex& \textbf{-0.170}**& 0.067& -4.02 \\
    Married& \textbf{-0.158}***& 0.039& -4.02 \\
    Education& \textbf{-0.130}***& 0.034& -3.81 \\
    Any PRL Use& \textbf{0.289}***& 0.029& 9.98 \\
    Tranquilizer Use& \textbf{0.114}***& 0.028& 4.14 \\
    Heroin Use& \textbf{0.244}***& 0.057& 4.24 \\
    Cocaine Use& \textbf{0.102}**& 0.04& 9.95 \\
    Amphetamines& \textbf{0.231}***& 0.036& 6.36 \\
    \bottomrule
    Significance level:& **0.05,& ***0.01&
  \end{tabular}
\end{table}


%%%%%%%%%%%%%%%%%%%%%%%%%%%%%%%%%%%%%%%%%%%%%%%%%%%%%%%%%%%%%%%%%%%%%%%%%%%%%%%%


Figure 4 shows the diagnostic plots of residuals from 
the multiple regression. The QQ plot in the upper right panel indicates that 
the dataset is approximately normal, but there may be a non-linear trend 
in the data. 




%%%%%%%%%%%%%%%%%%%%%%%%%%%%%%%%%%%%%%%%%%%%%%%%%%%%%%%%%%%%%%%%%%%%%%%%%%%%%%%%

\begin{table*}[ht]
  \caption{Comparison of OLS Parameter Estimates for Regression of Pain Reliever 
  Misuse and Abuse by Age Group (with Standardized Estimates)}
  \label{tab:freq}
  \begin{tabular}{llllll}
    \toprule
      & & & Age Group& &  \\
    \midrule
    Variables & 12-17& 18-25& 26-35& 36-49& 50+  \\
    \midrule
    Intercept& 6.156***& 6.147***& 5.774***& 5.227***& 5.859*** \\
           & -& -& -& -& - \\
    Sex& 0.229& \textbf{-0.474}***& \textbf{-0.701}***& -0.019& \textbf{-0.433}***  \\
           & (0.045)& (-0.083)& (-0.115)& (-0.003)& (-0.070) \\
    Married& 0.127& \textbf{-0.300}***& -0.081& -0.057& -0.126  \\
           & (0.022)& (-0.096)& (-0.037)& (-0.024)& (-0.045) \\
    Education& 0& -0.065& -0.120& \textbf{-0.159}*& \textbf{-0.352}*** \\
           &        & (-0.020)& (-0.037)& (-0.049)& (-0.119) \\
    Any PRL Use& \textbf{0.184}**& \textbf{0.390}***& \textbf{0.367}***& \textbf{0.435}***& \textbf{0.406}***  \\
           & (0.090)& (0.171)& (0.153)& (0.169)& (0.153) \\
    Tranquilizer Use& \textbf{0.010}***& \textbf{0.159}***& \textbf{0.192}***& \textbf{0.187}***& \textbf{0.242}*** \\
           & (0.005)& (0.080)& (0.085)& (0.080)& (0.097) \\
    Heroin Use& 0.183& \textbf{0.196}***& \textbf{0.314}***& \textbf{0.446}***& 0.089  \\
           & (0.024)& (0.054)& (0.098)& (0.103)& (0.018) \\
    Cocaine Use& 0.227& 0.086& -0.073& -0.045& 0.142  \\
           & (0.064)& (0.032)& (-0.026)& (-0.015)& (0.048) \\
    Amphetamines& \textbf{0.306}***& \textbf{0.170}***& \textbf{0.201}***& 0.143& -0.083  \\
           & (0.120)& (0.063)& (0.066)& (0.043)& (-0.020) \\
    \midrule
    \textit{n}       & 685& 2259& 1506&  1380& 648 \\
    \textit{Percent} & 10.56& 34.87& 23.24& 21.3& 10 \\ 
    \textit{F-test}  & 4.50*** & 30.52***& 18.56***& 15.17***& 6.52***  \\ 
    \textit{R-square}& 0.0444& 0.0979& 0.0902& 0.0813& 0.0755 \\ 
    \textit{Adj. R-square}& 0.0346 & 0.0947& 0.0845& 0.076& 0.0639 \\
    \bottomrule
    Significance level:&  **0.05,& ***0.01& &
  \end{tabular}
\end{table*}





%%%%%%%%%%%%%%%%%%%%%%%%%%%%%%%%%%%%%%%%%%%%%%%%%%%%%%%%%%%%%%%%%%%%%%%%%%%%%%%%
\section{Discussion}

The results showed that prescription opioid use, misuse, and abuse in this 
sample was higher than use of illicit opioids such as heroin and fentanyl. 
The use of Hydrocodone (Vicodan) was double that of Oxycodone (Oxycodone) 
across almost all age groups. Illicit drug use was highest between the ages 
of 18 to 25. Almost twice as many young adults reported a need for substance 
use treatment, but had not received treatment, compared to the youngest age 
group. Of individuals who reported misusing prescription opioid medications, 
twice as many said they had used heroin than had not (see Figure 1), which is 
consistent with the hypothesis that prescription opioid misuse is associated 
with heroin use. The different learning models provided different estimates 
of the features important for predicting pain reliever misuse and abuse. 
The multiple regression showed multiple features together significantly 
predicted pain reliever abuse, but there may be non-linear trends in the data.
Compared to ridge regression, the lasso performed variable selection, 
indicating that a model with five features (Treatment, Heroin, Cocaine, 
Amphetamine, Any Pain Relievers) explained a significant portion of
variabilty in pain reliever abuse, and adding more factors did not improve
the performance greatly. The regression tree model created a solution
with four features (Treatment, Cocaine, Heroin, Tranquilizers), and the
random forest regression selected Tranquilizers as most informative of
pain reliever misuse, which is consistent with the results of the gradient
boosting model tree ensemble (shown in Table 3). Overall, the models 
agreed in selecting the top five most influential predictors of pain
reliever abuse, but there was some variation in the ordering of importance. 
The linear models showed that substance treatment had the highest
coefficients (followed by heroin use), whereas the tree ensemble methods
indicated that tranquilizer use was the most important feature. Given the 
relatively low rates of prescription opioid and heroin use in the sample, 
additional evidence is needed to clarify questions regarding the importance
of various features for predicting opioid misuse and abuse. 
 

Theories of addiction suggest that situational cues or events can elicit a 
motivational state underlying relapse, as addictive behavior are reinstated by 
exposure to drugs, drug-related cues, or environmental stressors \cite{shaham03}. 


%%%%%%%%%%%%%%%%%%%%%%%%%%%%%%%%%%%%%%%%%%%%%%%%%%%%%%%%%%%%%%%%%%%%%%%%%%%%%%%%
\subsection{Limitations}
Survey research 
provides data on a wide range of issues that people may be reluctant to 
disclose, including mental health disorders, personal medical issues, 
prescription medications, and illicit drug use. Responses to surveys may be 
biased to some degree, and it can be difficult to obtain reliable information 
about illicit or illegal behaviors, but the anonymity of survey response are 
designed to preserve confidentiality and can help to assure more accurate 
disclosures. 


There may be some misspecification in our model as the public NSDUH data did not include ethnicity and geographical region which past research suggests are relevant to prescription opioid misuse [11]. Although employment status may contribute to pain reliever misuse, we excluded the variable ‘employment status 18’ from our data set as it did not include data for respondents between 12-17 years; past studies indicate that approximately 22% of high school seniors in the U.S. had some lifetime exposure to prescription opioids [12]. We also initially hoped to analyze trends in prescription opioids misuse over time, but were able to obtain data from NSDUH for only two years, and pooled data from 2015 and 2016 in cross-sectional data set. The NSDUH includes data from 1979-2016; however, given the large number of variables, and changes in the survey format across versions, constructing a panel data data set with aggregated variables across multiple years exceeded given time constraints for the project.  



A limitation of the study is that only a small proportion of the sample 
reported having used or misused prescription opioids.

Survey data can be biased by under-reporting or by 
minimizing reports of illicit substance use. People may also be reluctant to 
disclose mental health issues or health problems (e.g., STDs, HIV, suicide 
attempts). Another consideration is that, owing to the nature of the methods 
used in the study, it was not possible to determine the direction of the 
relationship between prescription opioid use and heroin use. Individuals who 
misuse or abuse prescription opioid medications may turn to synthetic opioids 
or heroin as cheaper, more readily available than prescription medications. 
Alternatively, individuals who used heroin may be inclined to abuse 
prescription opioids based on availability. Past research has shown that drug 
dosage and medication type play a significant role in opioid misuse: Treatment 
with high daily dose opioids (e.g., more than 120 mg/ day) and short-acting 
schedule II opioids increases the risk of misuse, abuse, and drug overdose 
\cite{sullivan10}. The results of the present analysis showed that demographic 
characteristics played a relatively minor role compared to the use of illicit 
drugs or tranquilizers. 

%%%%%%%%%%%%%%%%%%%%%%%%%%%%%%%%%%%%%%%%%%%%%%%%%%%%%%%%%%%%%%%%%%%%%%%%%%%%%%%%
\section{Conclusion}

A general conclusion is that people who reported misusing prescription opioids 
were also likely to have received substance treatment. More than any other 
demographic features, a history of prescription medication use, or illicit drug 
use, both seemed to contribute highly to the abuse of pain reliever medications, 
particularly for those who reported using tranquilizers, heroin, cocaine, or 
amphetamines. The findings suggest that the opioid crisis may be driven by the 
widespread availability of prescription medications. Even for people with no 
previous history of mental health issues, exposure to highly addictive opioid 
medications puts people at risk for drug dependency and addiction. As mentioned 
in the introduction, even for those who have been in drug treatment programs, 
a lack of continuity in treatment can leave many people in recovery at risk 
for relapse or possible overdose as they are released back into environments 
associated with their drug use. The sharp increase in overdose deaths in the 
U.S. due to synthetic opioids (other than methadone) has coincided with the 
increased availability of illicitly manufactured fentanyl \cite{nida17}. 
Because the dosage levels and potency of illicit opioids are largely unknown, 
there is greater risk of drug overdose death. Recent findings suggest the 
opioid overdose epidemic is getting worse, and requires urgent action to prevent 
opioid abuse, addiction, and death. The findings reported here seek to raise 
awareness about the risk factors for prescription opioid addiction for patients 
and health care providers in order to help reduce opioid overdose deaths. 

%%%%%%%%%%%%%%%%%%%%%%%%%%%%%%%%%%%%%%%%%%%%%%%%%%%%%%%%%%%%%%%%%%%%%%%%%%%%%%%%

\appendix

\section{Detection and WLS Correction for Heteroscedasticity}

There was some concern for heteroscedasticity in our data because of the 
cross-sectional variables, and also due to the non-normal distributions of 
scores in aggregated variables. An auxiliary regression was conducted for 
the White’s test yielded an =0.096, with N=24963 and 121 parameters, 
providing a  = 2401.44, greater than the critical value of (df=100)=124.34, 
at the alpha = 0.05; the null hypothesis was rejected, indicating evidence 
of strong heteroscedasticity.  The Goldfield-Quandt Test was conducted 
using C = 2952 (16 parameters), which resulted in a significant F-test of
1.24 which was greater than the FCRITICAL(500,200)= 1.22, at the alpha = 0.05 
level; the null hypothesis was also rejected indicating evidence of
heteroscedasticity. To correct for heteroscedasticity, we used the 
Weighted Least Squares (WLS) transformations based on the assumption 
that 2 = 2 ((AGECAT)2), with PROC MODEL in SAS to derive WLS regression 
equation and White’s test for this new model. The White’s test obtained 
from PROC MODEL WLS regression corrected for heteroscedasticity yielded 
a  = 2081, with p-value < 0.0001, which indicated that heteroscedasticity 
was reduced but not eliminated from the data.

The same procedure was conducted with the subset of MUPO. The auxiliary 
regression for the White’s test yielded an =0.064 (N=6478, 121 parameters), 
providing a  = 414.59, which was greater than the critical value of 
(df=100)=124.34, at the alpha = 0.05 level; the null hypothesis was 
rejected, indicating evidence of heteroscedasticity.  The Goldfield-Quandt 
Test was conducted using C = 1505 (with 16 parameters), which resulted in 
a F- test of 1.10, which was less than FCRITICAL(500,200)= 1.22, at the 
alpha = 0.05 level; this non-significant result suggested there was not 
sufficient evidence of heteroscedasticity. However, we decided to use the
WLS transformations based on the assumption that: 2 = 2 ((AGECAT)2), 
with PROC MODEL in SAS to derive WLS regression parameter estimates. 



%%%%%%%%%%%%%%%%%%%%%%%%%%%%%%%%%%%%%%%%%%%%%%%%%%%%%%%%%%%%%%%%%%%%%%%%%%%%%%%%




%ACM-Reference-Format
\bibliographystyle{unsrt}
\bibliography{report} 


\end{document}
