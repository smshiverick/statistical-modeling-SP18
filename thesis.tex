\documentclass[sigconf]{acmart}

\input{format/final}

\begin{document}
\title{Modeling Opioid Pain Reliever Misuse and Abuse}

  \author{Sean M. Shiverick}
  \affiliation{%
  \institution{Indiana University Bloomington}
}
\email{smshiver@iu.edu}

\renewcommand{\shortauthors}{S.M. Shiverick}

%%%%%%%%%%%%%%%%%%%%%%%%%%%%%%%%%%%%%%%%%%%%%%%%%%%%%%%%%%%%%%%%%%%%%%%%%%%%%%%%

\begin{abstract}
This study examined demographic features, medications, and illicit drug use 
as determinants of pain reliever misuse and abuse. Data from the National 
Survey on Drug Use and Health (2015-16). Weighted least squares (WLS) regression
was conducted with a subset of N=6,964 respondents who reported misuse and abuse 
of prescription opioids. Age group, sex, marital status, and education
level were negatively related to pain reliever misuse and abuse, indicating that
some demographic features may be protective of opioid misuse. By contrast, use of 
medications and illicit drugs were positively related to pain reliever misuse 
and abuse. Use of any opioid pain relievers and tranquilizers was significantly
related to pain reliever misuse and abuse across age groups. Use of cocaine and 
amphetamines was more influential than heroin use in predicting pain reliever 
misuse and abuse; however, these effects varied by age group. Heroin use was 
significantly related to pain reliever misuse and abuse for individuals between 
the ages of 18 and 49 years. Use of amphetamines was significantly related to 
pain reliever misuse and abuse for individuals between 12 and 35 years of age. 
\end{abstract}

\keywords{Opioid Misuse, Pain Reliever Abuse, Linear Regression Models}

\maketitle

%%%%%%%%%%%%%%%%%%%%%%%%%%%%%%%%%%%%%%%%%%%%%%%%%%%%%%%%%%%%%%%%%%%%%%%%%%%%%%%%
\section{Introduction}

Over the past two decades, prescription opioid misuse, abuse, and addiction 
in the U.S. has become a major crisis with serious public health consequences.
In 2015, an estimated 2 million Americans suffered from substance use disorders 
related to opioid pain medications such as oxycodone and hydrocodone 
\cite{nida18,cdc18}. Of patients legitimately prescribed opioids for chronic 
pain, approximately 25\% misused them, between 8\% to 12\% became addicted, and 
4\% to 6\% transitioned to heroin \cite{vowles15, carlson16}. Opioid dependence 
and addiction are chronic health conditions; following treatment, many addicted 
individuals are at high risk for relapse and overdose death \cite{shaham03}. 
Since 1999, the number of overdose deaths from prescription opioids has more 
than quadrupled \cite{cdc16}. The economic cost of the opioid crisis since 2001 
is estimated to exceed one trillion dollars \cite{altarum18}. 

\subsection{Literature Review}

Past studies have identified risk factors for opioid abuse, including gender, 
ethnicity, comorbid psychological disorders, and non-opioid drug abuse 
\cite{yokell13,rice12}. Rates of hospitalization for prescription opiate overdose 
(POD) are higher for females than males, and the increase in POD is highest for
Whites compared to Blacks or Hispanics \cite{unick13}. Supply-based interventions 
to reduce the availability of prescription opioids have produced a shift to 
heroin use, and the exponential increase in POD and heroin overdose deaths (HOD) 
are correlated \cite{jones15,reifler12}. The non-medical use of prescription 
opioids may also vary by age \cite{mccabe12}. The present study modeled the 
relationships between demographic features, medications, and illicit drugs as 
predictors of pain reliever misuse and abuse using. Identifying factors which 
are negatively related with pain reliever misuse may reveal preventative 
characteristics that decrease opioid dependence and addiction. Models of pain 
reliever misuse and abuse are compared across age groups. 

%%%%%%%%%%%%%%%%%%%%%%%%%%%%%%%%%%%%%%%%%%%%%%%%%%%%%%%%%%%%%%%%%%%%%%%%%%%%%%%%

Many researchers have analyzed the probability that an individual abused 
or misused opioids (or did not) with logistic regression 
\cite{rice12, unick13, jones15, mccabe12}. The logistic model takes the form:
$ Pr(abuse=1) = F(\alpha+\sum_j(\beta_i*X_ij)) $, where the dependent variable
is a binary outcome. In the present study, pain reliever misuse and abuse was 
assessed along a continuum by aggregating responses for several related binary 
responses into a single measure. Similarly, responses for related independent 
measures were summed, creating several aggregated variables as regressors in a 
linear regression model rather than a logit model. The general linear model 
(Ordinary Least Squares) estimates the coefficients of the independent 
variables in relation the the dependent variable, with some degree of error. 

\begin{equation}
  \ Y_i = \beta_1*X_1 + \beta_2*X_2 +... + \beta_k*X_k + u_i
\end{equation}

\subsection{The Model}

The classical linear regression model is based on several assumptions, 
including: the model is linear in the parameters, the fixed x-values are 
independent of the error term, the mean value of the error terms is zero, 
non-collinearity among the independent variables, and independent 
observations (i.e., non-autocorrelation). An important assumption for the 
present study, is that the disturbances $u_i$ all have the same variances 
(i.e., \emph{homoscedasticity}) \citeN{gujarati09}. Heteroscedastic 
(\emph{non-equal}) variances is a common occurrence in socio-economic 
variables measured in cross-section, but can also arise due to skewness in 
the distribution of one or more regressors in the model, or from incorrect 
functional form (e.g., linear versus log-linear models). Outliers can also 
contribute to heteroscedasticity. If the assumption of homoscedasticity is 
not met, the ordinary least squares (OLS) parameter estimates (though still 
linear and unbiased) are no longer ``best'', are not efficient as they do 
not provide the minimum variance, and yield unreliable statistics. In the 
presence of heteroscedasticity, Weighted Least Squares (WLS)-a subset of 
Generalized Least Squares (GLS)-provides estimates with a smaller variance to 
replace the OLS estimators.  To control for heteroscedasticity using WLS, first 
the primary contributing variable(s) must be identified; then an appropriate 
transformation is carried out on the data to correct for heteroscedasicity. 
OLS is then applied to the transformed data; WLS provides `estimates which
are `best', `linear', `unbiased' (i.e., BLUE). A common ad hoc assumption 
with WLS is that the error variance is proportional to the square of one of 
the explanatory variables $\sigma^2=\sigma^2 X_ji^2$ \citeN{gujarati09}. 
The originaal model is transformed by dividing through by the explanatory 
variable $X_i$:

\begin{equation}
  \ Y_i/X_2 = \beta_1*X_2 + \beta_2 +... + \beta_k*X_k/X_2 + u_i/X_2
\end{equation}

%%%%%%%%%%%%%%%%%%%%%%%%%%%%%%%%%%%%%%%%%%%%%%%%%%%%%%%%%%%%%%%%%%%%%%%%%%%%%%%%

\subsection{The Data} 

The study data was derived from the National Survey on Drug Use and Health 
(NSDUH) from 2015 and 2016, obtained from the Substance Abuse and Mental Health 
Data Archive (SAMHDA) \cite{samhsa16}. The NSDUH is a large survey that consists
of approximately 2600 questions covering all aspects of substance use, misuse, 
dependency, abuse, and addiction for prescription medications and illicit drugs, 
with a comprehensive set of demographic questions related to physical health, 
depression, mental health treatment, and drug and alcohol treatment. According 
to the NSDUH codebook, sampling was weighted across states by population size 
for a representative distribution selected from 6,000 area segments. The sample 
design used five state sample size groups, drawing more heavily from the eight 
states with the largest population (e.g., CA, FL, IL, MI, NY, OH, PA, TX) which 
together account for 48 percent of the total U.S. population aged 12 or older.  
All identifying information in the public data set were collapsed (e.g., age 
categories) and state identifiers were removed from the public use file to 
ensure confidentiality. The NSDUH public-use files do not include geographic 
location, or demographic variables related to ethnicity or immigration status. 
The weighted survey screening response rate was 81.94 percent and the weighted 
interview response rate was 71.2 percent. For the present study, approximately
90 variables were selected that were aggregated to a data set of 20 variables. 


%%%%%%%%%%%%%%%%%%%%%%%%%%%%%%%%%%%%%%%%%%%%%%%%%%%%%%%%%%%%%%%%%%%%%%%%%%%%%%%%
\subsection{Study Goals} 

The main goal was to assess demographic and environmental features as 
determinants of opioid pain reliever misuse and abuse. It was hypothesized 
that regressors negatively associated with the dependent variable may play a 
preventive role in decreasing opioid misuse and abuse. Past research indicates
that non-medical use and misuse of prescription opioids varies by age, and 
age group was considered an important covariant in the model that may 
contribute to heteroscedasticity. WLS transformations were used to correct for
heteroscedasticity due to age categories. It was predicted that pain reliever 
misuse and abuse was expected to decrease with age. Use of non-opiod medications
and illicit drug was expected to predict misuse of prescription opioids. It
was expected that heroinuse would be highly correlated with pain reliever 
misuse and abuse. The analysis also examined the subset of individuals who 
reported ever misusing and abusing pain relievers in the past. Finally, separate 
regression models were constructed for each age group to examine differences 
in the relative influence of predictors across different age subsets. 

%%%%%%%%%%%%%%%%%%%%%%%%%%%%%%%%%%%%%%%%%%%%%%%%%%%%%%%%%%%%%%%%%%%%%%%%%%%%%%%%
\section{Method}

The following steps were included in the project workflow pipeline: 
(1) Download and extract the data, (2) Data cleaning and preparation, 
(3) Exploratory data analysis, (4) Data visualization, (5) Analysis of 
learning models for opioid pain reliever misuse and abuse.

\subsection{NSDUH Data Set} 

NSDUH data from 2015 and 2016 was downloaded from the SAMHDA website 
\cite{samhsa16} URL using Python. The data files were extracted as a data 
frame object using Pandas, and the data was written to CSV file for 
examination. The dataset consists of 57,146 observations with 2,666 features 
representing individual-level responses from a survey of the U.S. population. 

The main dependent 
variables was aggregated measure of MUPO as the dependent variable based on the self-reported non-medical use of prescription opioids not directed by a physician, use of pain relievers in amounts greater than prescribed, misuse of pain relievers in the past year or month, dependence or abuse of pain relievers, and use of pain relievers to get “high”. 

The NSDUH is large scale, comprehensive survey with 2,666 variables; a sample of approximately 90 variables was selected, including demographic information (age category, sex, marital status, education level, employment, size of metropolitan area). Responses for related discrete variables (i.e., No=0 / Yes=1) were summed to create aggregated variables, for any health problems ever: 'HEALTH'  = (STD + Hepatitis + HIV + Cancer + Hospitalized). Similarly, and overall mental health (`MENTHLTH`) included adult major depressive episode, emotional distress, suicidal thoughts, or suicidal plans. 
The dependent variable, pain reliever misuse and abuse, (`PRLMISAB`) included responses to prescription opioid pain misuse, abuse, dependency, and past year use, misuse or abuse. Any prescription opioids medications taken in past year (e.g., Hydrocodone, Oxycodone, Tramadol, Morphine, Fentanyl, Oxymorphone, Demerol, Hydromorphone), were summed to create a single variable (`PRLANY`). Only respondents who indicated past opioid pain reliever use were selected for inclusion in the sample; 13,916 respondents from the NSDUH-2015, 11690 respondents from the NSHUH-2016 were merged to create a single data set with a total N=25,606 respondents, as pooled, cross-sectional data. Aggregated variables were created for past use of: Tranquilizers, Sedatives, Heroin, Cocaine, Amphetamines, Hallucinogens, Drug Treatment, and Mental Health Treatment, resulting in a total of 16 variables [14]. 







\begin{table*}[ht]
  \caption{Summary of Variables in Initial Data Set (16 Independent Variables)}
  \label{tab:freq}
  \begin{tabular}{ll}
    \toprule
    \textit{Dependent Variable} & Label \\
    \midrule
    Prescription Opioid Pain Reliever Misuse and Abuse (0-12 Scale)& PRLMISAB  \\
    \midrule
    \textit{Demographic Variables}&   \\
    \midrule
    Age Category (1=12-17 years, 2=18-25, 3=26-34, 4=35-49, 5=50 and older)& AGECAT \\
    Biological Sex (0=Male, 1=Female)& SEX  \\
    Marital Status (0=Unmarried, 1=Divorced, 2=Widowed, 3=Married)& MARRIED  \\
    Education (1=Less H.S., 2=H.S. Grad., 3=Some College,  4=College Grad.)& EDUCAT  \\
    Size of City/Metropolitan Region (1=Rural, 2=Small, 3=Large)& CTYMETRO  \\
    Health Problems Aggregated  (0-10 scale)& HEALTH  \\
    Mental Health, Aggregated: adult depression, emotional distress (0-10 scale)& MENTHLTH  \\
    Treatment for Drugs and Alcohol in past year, Aggregated (0-5 scale)& TRTMENT  \\
    Mental Health Treatment, Aggregated (Likert scale, 1-10)& MHTRTMT  \\
    \midrule
    \textit{Medication and Drug Use Variables}& \\
    \midrule
    Any Prescribed Opioid Pain Reliever Use in past year, Aggregated (1-10 scale)& PRLANY  \\
    Tranquilizer use, past year, Aggregated (Likert scale, 0-5)& TRQLZRS \\
    Sedative use, past year, Aggregated (0-5 scale)& SEDATVS  \\
    Heroin use, past year, Aggregated (0-5 scale)& HEROINUSE  \\
    Cocaine and Crack Cocaine Use in past year, Aggregated  (0-5 scale)& COCAINE  \\
    Amphetamine and Methamphetamine Use in past year, Aggregated (0-5 scale)& AMPHETMN  \\
    Hallucinogen Use in past year, Aggregated (0-5 scale)& HALUCNG \\
    \bottomrule
  \end{tabular}
\end{table*}





%%%%%%%%%%%%%%%%%%%%%%%%%%%%%%%%%%%%%%%%%%%%%%%%%%%%%%%%%%%%%%%%%%%%%%%%%%%%%%%%
\subsection{Data Cleaning and Preparation }

\subsubsection{Data Cleaning}
All steps of this analysis were completed in a python interactive notebook 
based on examples from \emph{Python for Data Analysis} \cite{mckinney17}. 
After saving the NSDUH-2015 as a data frame object, the dataset was subset 
by columns to include demographic characteristics (e.g., age category, sex, 
marital status, education, employment status, and category of metropolitan 
area), measures of physical health (e.g., overall health, STDs, Hepatitis, 
HIV, Cancer, hospitalization), mental health (e.g., Adult Depression, severe
Emotional Distress, Suicidal Thoughts, Plans), Pain Reliever (`PRL`) 
Medication Use, Misuse, and Abuse (over past year, past month), Prescription 
Opioid Medications Taken in Past year (e.g., Hydrocodone, Oxycodone, Tramadol, 
Morphine, Fentanyl, Oxymorphone, Demerol, Hydromorphone), Heroin Use, Abuse 
(over past year, past month), Tranquilizer Use, Sedative Use, Cocaine Use, 
Amphetamine and Methamphetamine Use, Hallucinogen Use, Drug Treatment (e.g., 
Inpatient, Outpatient, Hospital, Mental Health Clinic, ER, Drug Treatment 
Status), and Mental Health Treatment History. A codebook was created to 
provide a complete list of variables included with summaries of response 
categories. The following steps were taken to detect and remove 
inconsistencies in the data:
\begin{enumerate}
  \item Remove missing values (i.e., NaN) 
  \item Recode blanks, non-responses, or legitimate skips (e.g., 99, 991, 
  993) to zero  
  \item Recode dichotomous responses (e.g., Yes=1 / No=2) so that No=0
  \item Recode categorical variables to be consistent with amount or degree 
  (e.g., 1=low, 2=med, 3=high)
   \item Rename selected variables for better description (e.g., 
   Adult Major Depressive Episode Lifetime changed from AMDELT to DEPMELT)
\end{enumerate}


%%%%%%%%%%%%%%%%%%%%%%%%%%%%%%%%%%%%%%%%%%%%%%%%%%%%%%%%%%%%%%%%%%%%%%%%%%%%%%%%
\subsubsection{Aggregated Variables}

The majority of features were represented as dichotomous Yes / No variables; 
therefore, related features were summed to create aggregated variables. 
For example, overall health, STDs, Hepatitis, HIV, Cancer, and overnight
hospitalization, were aggregated to create a single measure, 'HEALTH'. The 
health measure was recoded so that higher scores indicated better health. 
Questions related to adult major depressive episode, emotional distress, 
suicidal thoughts, and suicidal plans, were summed to create a single mental 
health variable (`MENTHLTH`) with scores ranging from 0 to 9. Responses to 
pain reliever (`PRL`) medication use, misuse, abuse, or dependency, were 
aggregated to create a single variable of pain reliever misuse or abuse 
(`PRLMISAB`). Any prescription painkiller medications used in the past year 
were summed into a single variable (`PRLANY`). Similarly, all related responses 
were summed to create single variables for Tranquilizers, Sedatives, Cocaine, 
Amphetamines, Hallucinogens, Drug Treatment, and Mental Health Treatment. 
The target outcome was any prescription opioid pain reliever medication misuse 
or abuse, `PRLMISAB`. The demographic characteristics and aggregated variables 
were subset and saved to a new data frame consisting of 22 features and 57,146 
observations, which was exported to CSV file entitled `project-data.csv`. 





%%%%%%%%%%%%%%%%%%%%%%%%%%%%%%%%%%%%%%%%%%%%%%%%%%%%%%%%%%%%%%%%%%%%%%%%%%%%%%%%
\section{Results}

\subsection{Exploratory Data Analysis}

The initial sample consisted of 25,606 observations. Preliminary examination 
of the data revealed 174 outliers that exceeded the range of the dependent 
variable and were excluded, resulting in a sample of N=25,432 (10655 male, 
14777 female). Overall, the majority of individuals (73\%) reported never 
misusing or abusing prescription opioids; a subset of N=6,964 individuals 
reported misusing prescription opioids (i.e., "MUPO") at some point. Figure 1 
shows that the proportion of pain reliever misuse or abuse was highest for 
individuals between 18 to 25 years of age. In general, pain reliever misuse 
and abuse was higher among younger than older age groups. More women (61\%)
than men (39\%) reported using any opioid pain relievers, but equal proportions 
of men and women (50\%) reported pain reliever misuse and abuse. Table 2 
provides summary statistics of aggregated variables measured on a scale for 
both the full sample and the MUPO subset. The mean and standard deviation for 
the dependent variable are more robust in the MUPO subset than the full sample.  

\begin{table*}[ht]
  \caption{Summary Statistics for Aggregated Variables for Full Sample and 
  Subset of Individuals Reporting Misuse and Abuse of Prescription Opioids (MUPO)}
  \label{tab:freq}
  \begin{tabular}{llllllll}
    \toprule
     & Full Sample& (N = 25,432)&&& MUPO Subset& (N = 6,946)&  \\
    \midrule
    Variables & Mean& S.D.& Median& Range& Mean& S.D.& Median  \\
    \midrule
    PRL Misuse Abuse& 1.74& 3.30& 0& 12& 6.38& 3.11 & 7 \\
    Health Problems& 2.77& 1.15& 3& 7& 2.71& 1.11& 3  \\
    Mental Health& 1.57& 2.37& 0& 12& 2.21& 2.71& 1  \\
    Drug Treatment& 0.15& 0.89& 0& 10& 0.43& 1.48& 0  \\
    MH Treatment& 0.37& 0.86& 0& 7& 0.48& 0.97& 0  \\
    Any PRL Use& 1.61& 1.06& 1& 10& 1.93& 1.37& 1 \\
    Tranquilizer Use& 0.60& 1.22& 0& 5& 1.05& 1.42& 0  \\
    Sedative Use& 0.12& 0.50& 0& 5& 0.18& 0.62& 0  \\
    Heroin Use& 0.10& 0.49& 0& 5& 0.30& 0.85& 0  \\
    Cocaine use& 0.38& 0.80& 0& 5& 0.83& 1.10& 0 \\
    Amphetamines& 0.32& 0.72& 0& 5& 0.73& 1.04& 0  \\
    Hallucinogens& 0.65& 1.16& 0& 5& 1.42& 1.47& 1  \\
    \bottomrule
  \end{tabular}
\end{table*}


Figure 1 shows the proportion of individuals who reported misusing prescription 
opioid pain relievers and also using heroin. The left column of the Figure 1 
shows the majority of respondents (89 percent) stated they had never misused 
opioid pain medication or used heroin, although 10 percent reported misusing 
opioid pain medication at some point. The right panel of Figure 1 shows that, 
of those individuals who reported using heroin, the proportion who reported 
misusing opioid pain medication was roughly twice as large as the proportion 
who reported only using heroin. This is consistent with the hypothesized 
link misuse of prescription opioids and heroin use.

\begin{figure}[!ht]
  \centering\includegraphics[width=\columnwidth]{images/Figure1.pdf}
  \caption{Proportion of Individuals Reporting Opioid Pain Reliever Misuse 
  or Abuse (PRLMA) by Age Group}
  \label{f:Figure1}
\end{figure}

%%%%%%%%%%%%%%%%%%%%%%%%%%%%%%%%%%%%%%%%%%%%%%%%%%%%%%%%%%%%%%%%%%%%%%%%%%%%%%%%

Figure 2 shows the aggregated measure of Opioid Pain Reliever misuse and abuse 
plotted against the aggregated measure of Heroin use (which includes misuse, 
abuse, lifetime use, past year use, 30 day use), with weighted regression 
lines grouped by size of City Metropolitan region (from none to large). 
The largest proportion of the sample who report prescription opioid misuse, 
abuse, and heroin use is represented by observations from large metropolitan 
areas (red circles) with large population size. However, a small number of
observations from rural or small metropolitan regions (blue and green circles)
showed very high rates of prescription opioid misuse and abuse. Regression 
lines (i.e., line of best fit) shown are weighted by the City/Metro
region attribute, with a steeper slope shown for smaller metropolitan regions 
than large metropolitan regions. The difference in slope may be due to the 
influence of the small number of outliers who had high degrees of prescription 
opioid misuse, and heroin use. The plot also shows a clear divide on the y-axis,
which separates the sample according to high and low or no prescription opioid 
misuse, although the continuum of heroin use from no, low, to high is 
distributed fairly evenly along the x-axis. 

\begin{figure}[!ht]
  \centering\includegraphics[width=\columnwidth]{images/Figure2.pdf}
  \caption{Proportion of Males and Females Who Report Opioid Pain Misuse 
  and Abuse}
  \label{f:Figure2}
\end{figure}

%%%%%%%%%%%%%%%%%%%%%%%%%%%%%%%%%%%%%%%%%%%%%%%%%%%%%%%%%%%%%%%%%%%%%%%%%%%%%%%%

Figure 3 

; only 956 respondents had used heroin (570 males, 
386 females). 
 A very 
small proportion of the entire sample reported both misusing and abusing 
prescription opioids and using heroin, but this is a group of interest. The 
last column of the second row shows the individuals reporting high levels of 
opioid misuse and abuse were distributed evenly across city/metropolitan areas 
of different sizes, with only slightly higher numbers for small cities or 
rural areas. As stated above, only few participants reported using heroin, and 
of these, the majority were from large metropolitan areas. Finally, the sample 
seems to have slightly higher proportions from small and large metropolitan 
areas, which is likely due to weighted sampling, which drew more from heavily 
populated regions.

\begin{figure}[!ht]
  \centering\includegraphics[width=\columnwidth]{images/Figure3.pdf}
  \caption{Pairplots of Mental Health, Prescription Opioid Misuse and Abuse,
  Heroin Use, and Size of City Metropolitan Areaand Proportion Who Reported Using Heroin}
  \label{f:Figure3}
\end{figure}


shows the pairplots of demographic features including mental health
(higher scores equal to more depression), Prescription Opioid Pain Reliever
(PRL) Medication (aggregated), Heroin Use (aggregated measure), and Size of 
City/Metropolitan region. The top row shows that the majority of the sample 
reported no mental health concerns, whereas a small proportion of the sample 
reported depression, emotional distress, or suicidal thoughts. Only few people 
self-described as high in depression reported low Prescription Opioid PRL 
misuse and abuse. The plot also reveals that prescription opioid misuse and 
heroin use were distributed approximately evenly for individuals reporting 
either low, moderate, or high levels of depression, which suggests that 
depression was not a factor in predicting opioid misuse. The second row shows 
a small number of individuals from rural areas or small cities who reported 
very high levels of prescription opioid misuse, although the majority of 
respondents misusing or abusing prescription opioid were from large 
metropolitan areas. As described above, the majority of respondents (about
90 percent of the sample) reported they had never misused prescription 
opioids. In the second row and third and fourth columns, a natural break is 
seen between individuals who reported high levels of prescription opioid 
misuse and abuse and those who reported very low or no opioid misuse.

%\begin{figure}[!ht]
%  \centering\includegraphics[width=\columnwidth]{images/Figure4.pdf}
%  \caption{Diagnostic Plots of Residuals for the Multiple Regression of
%  Prescription Opioid Pain Reliever Misuse and Abuse}
%  \label{f:Figure4}
%\end{figure}




%%%%%%%%%%%%%%%%%%%%%%%%%%%%%%%%%%%%%%%%%%%%%%%%%%%%%%%%%%%%%%%%%%%%%%%%%%%%%%%%
\subsection{Weighted Least Squares (WLS) Regression Models}

After loading the project data into R (with 57146 rows, 
22 columns), some additional data cleaning was performed to prepare the 
dataset for analysis. The ID column was deleted, two binomial variables were 
removed (e.g, PRLMISEVR, HEROINEVR), and frequency of heroin use (HEROINFQY) 
was excluded because it represented very few participants and was highly skewed. 
The dataset was split into the training set and test sets. A general linear 
model was fit with opioid pain relievers misuse and abuse (PRLMISAB) as the 
target variable, and with the following 18 variables entered as predictors: 
AGECAT, SEX, MARRIED, EDUCAT, EMPLOYED, CTYMETRO, HEALTH, MENTHLTH, SUICATT,
PRLANY, HEROINUSE, TRQLZRS, SEDATVS, COCAINE, AMPHETMN, TRTMENT, MHTRTMT. 
The results of the regression revealed that AGECAT, SEX, EMPLOYED, HEALTH, 
MNTLHLTH, PRLANY, HEROINUSE, TRQLZRS, SDATVS, COCAINE, AMPHETMN, and TRTMENT, 
together significantly predicted opioid pain reliever misuse and abuse 
PRLMISAB (p < 0.001). Figure 4 shows the diagnostic plots of residuals from 
the multiple regression. The QQ plot in the upper right panel indicates that 
the dataset is approximately normal, but there may be a non-linear trend 
in the data. 


%%%%%%%%%%%%%%%%%%%%%%%%%%%%%%%%%%%%%%%%%%%%%%%%%%%%%%%%%%%%%%%%%%%%%%%%%%%%%%%%

\begin{table*}[ht]
  \caption{Weighted Least Squares (WLS) Parameter Estimates for Regression 
  of Pain Reliever Misuse and Abuse}
  \label{tab:freq}
  \begin{tabular}{lllllll}
    \toprule
     & Full Sample& (N = 25,432)&& MUPO Subset& (N = 6,946)&  \\
    \midrule
    Variables & Estimate& S.E.& t-value& Estimate& S.E.& t-Value  \\
    \midrule
    Intercept& 1.656***& 0.083& 19.85& 6.870***& 0.154& 44.72 \\
    Age Category& \textbf{-0.370}& 0.028& -13.23& \textbf{-0.312}***& 0.059& -5.31 \\
    Sex& \textbf{-0.156}***& 0.039& -4.03& \textbf{-0.154}***& 0.069& -2.24  \\
    Married& \textbf{-0.041}**& 0.021& -2.02& \textbf{-0.159}***& 0.039& -4.03  \\
    Education& -0.028& 0.020& -1.41& \textbf{-0.123}***& 0.037& -3.36 \\
    City/Metro Size& -0.016& 0.025& -0.67& 0.017& 0.043& 0.40 \\
    Health Problems& -0.021& 0.019& -1.12& -0.041& 0.034& -1.22 \\
    Mental Health& \textbf{0.059}***& 0.011& 5.22& -0.022& 0.018& -1.25 \\
    Drug Treatment& \textbf{0.190}***& 0.025& 7.61& -0.13& 0.024& -0.57 \\
    MH Treatment& \textbf{-0.143}***& 0.033& -4.39& 0.017& 0.029& 0.33 \\
    Any PRL Use& \textbf{0.252}***& 0.020& 12.62& \textbf{0.290}***& 0.029& 9.87 \\
    Tranquilizer Use& \textbf{0.447}***& 0.021& 21.40& \textbf{0.116}***& 0.029& 4.05 \\
    Sedative Use& \textbf{0.192}***& 0.039& 4.95& 0.043& 0.053& 0.81 \\
    Heroin Use& \textbf{0.358}***& 0.055& 6.52& \textbf{0.258}***& 0.059& 4.36 \\
    Cocaine Use& \textbf{0.350}***& 0.035& 9.95& \textbf{0.103}***& 0.043& 2.38 \\
    Amphetamines& \textbf{0.780}***& 0.030& 25.93& \textbf{0.232}***& 0.037& 6.23 \\
    Hallucinogens& \textbf{0.576}***& 0.023& 25.66& 0.006& 0.030& 0.21 \\
    \bottomrule
    Significance level:& **0.05,& ***0.01&&&&
  \end{tabular}
\end{table*}


\begin{table}
  \caption{WLS Parameter Estimates for Final Model of Pain Reliever Misuse and Abuse 
  with MUPO Subset (N=6964)}
  \label{tab:freq}
  \begin{tabular}{llll}
    \toprule
    Variable & Estimate& S.E.& t-Statistic \\
    \midrule
    Intercept& 6.850***& 0.098& 69.69 \\
    Age Category& \textbf{-0.327}***& 0.058& -5.64 \\
    Sex& \textbf{-0.170}**& 0.067& -4.02 \\
    Married& \textbf{-0.158}***& 0.039& -4.02 \\
    Education& \textbf{-0.130}***& 0.034& -3.81 \\
    Any PRL Use& \textbf{0.289}***& 0.029& 9.98 \\
    Tranquilizer Use& \textbf{0.114}***& 0.028& 4.14 \\
    Heroin Use& \textbf{0.244}***& 0.057& 4.24 \\
    Cocaine Use& \textbf{0.102}**& 0.04& 9.95 \\
    Amphetamines& \textbf{0.231}***& 0.036& 6.36 \\
    \bottomrule
    Significance level:& **0.05,& ***0.01&
  \end{tabular}
\end{table}






%%%%%%%%%%%%%%%%%%%%%%%%%%%%%%%%%%%%%%%%%%%%%%%%%%%%%%%%%%%%%%%%%%%%%%%%%%%%%%%%

\begin{table*}[ht]
  \caption{Comparison of OLS Parameter Estimates for Regression of Pain Reliever 
  Misuse and Abuse by Age Group (with Standardized Estimates)}
  \label{tab:freq}
  \begin{tabular}{llllll}
    \toprule
      & & & Age Group& &  \\
    \midrule
    Variables & 12-17& 18-25& 26-35& 36-49& 50+  \\
    \midrule
    Intercept& 6.156***& 6.147***& 5.774***& 5.227***& 5.859*** \\
           & -& -& -& -& - \\
    Sex& 0.229& \textbf{-0.474}***& \textbf{-0.701}***& -0.019& \textbf{-0.433}***  \\
           & (0.045)& (-0.083)& (-0.115)& (-0.003)& (-0.070) \\
    Married& 0.127& \textbf{-0.300}***& -0.081& -0.057& -0.126  \\
           & (0.022)& (-0.096)& (-0.037)& (-0.024)& (-0.045) \\
    Education& 0& -0.065& -0.120& \textbf{-0.159}*& \textbf{-0.352}*** \\
           &        & (-0.020)& (-0.037)& (-0.049)& (-0.119) \\
    Any PRL Use& \textbf{0.184}**& \textbf{0.390}***& \textbf{0.367}***& \textbf{0.435}***& \textbf{0.406}***  \\
           & (0.090)& (0.171)& (0.153)& (0.169)& (0.153) \\
    Tranquilizer Use& \textbf{0.010}***& \textbf{0.159}***& \textbf{0.192}***& \textbf{0.187}***& \textbf{0.242}*** \\
           & (0.005)& (0.080)& (0.085)& (0.080)& (0.097) \\
    Heroin Use& 0.183& \textbf{0.196}***& \textbf{0.314}***& \textbf{0.446}***& 0.089  \\
           & (0.024)& (0.054)& (0.098)& (0.103)& (0.018) \\
    Cocaine Use& 0.227& 0.086& -0.073& -0.045& 0.142  \\
           & (0.064)& (0.032)& (-0.026)& (-0.015)& (0.048) \\
    Amphetamines& \textbf{0.306}***& \textbf{0.170}***& \textbf{0.201}***& 0.143& -0.083  \\
           & (0.120)& (0.063)& (0.066)& (0.043)& (-0.020) \\
    \midrule
    \textit{n}       & 685& 2259& 1506&  1380& 648 \\
    \textit{Percent} & 10.56& 34.87& 23.24& 21.3& 10 \\ 
    \textit{F-test}  & 4.50*** & 30.52***& 18.56***& 15.17***& 6.52***  \\ 
    \textit{R-square}& 0.0444& 0.0979& 0.0902& 0.0813& 0.0755 \\ 
    \textit{Adj. R-square}& 0.0346 & 0.0947& 0.0845& 0.076& 0.0639 \\
    \bottomrule
    Significance level:&  **0.05,& ***0.01& &
  \end{tabular}
\end{table*}





%%%%%%%%%%%%%%%%%%%%%%%%%%%%%%%%%%%%%%%%%%%%%%%%%%%%%%%%%%%%%%%%%%%%%%%%%%%%%%%%
\section{Discussion}

The results showed that prescription opioid use, misuse, and abuse in this 
sample was higher than use of illicit opioids such as heroin and fentanyl. 
The use of Hydrocodone (Vicodan) was double that of Oxycodone (Oxycodone) 
across almost all age groups. Illicit drug use was highest between the ages 
of 18 to 25. Almost twice as many young adults reported a need for substance 
use treatment, but had not received treatment, compared to the youngest age 
group. Of individuals who reported misusing prescription opioid medications, 
twice as many said they had used heroin than had not (see Figure 1), which is 
consistent with the hypothesis that prescription opioid misuse is associated 
with heroin use. The different learning models provided different estimates 
of the features important for predicting pain reliever misuse and abuse. 
The multiple regression showed multiple features together significantly 
predicted pain reliever abuse, but there may be non-linear trends in the data.
Compared to ridge regression, the lasso performed variable selection, 
indicating that a model with five features (Treatment, Heroin, Cocaine, 
Amphetamine, Any Pain Relievers) explained a significant portion of
variabilty in pain reliever abuse, and adding more factors did not improve
the performance greatly. The regression tree model created a solution
with four features (Treatment, Cocaine, Heroin, Tranquilizers), and the
random forest regression selected Tranquilizers as most informative of
pain reliever misuse, which is consistent with the results of the gradient
boosting model tree ensemble (shown in Table 3). Overall, the models 
agreed in selecting the top five most influential predictors of pain
reliever abuse, but there was some variation in the ordering of importance. 
The linear models showed that substance treatment had the highest
coefficients (followed by heroin use), whereas the tree ensemble methods
indicated that tranquilizer use was the most important feature. Given the 
relatively low rates of prescription opioid and heroin use in the sample, 
additional evidence is needed to clarify questions regarding the importance
of various features for predicting opioid misuse and abuse. 
 

Theories of addiction suggest that situational cues or events can elicit a 
motivational state underlying relapse, as addictive behavior are reinstated by 
exposure to drugs, drug-related cues, or environmental stressors \cite{shaham03}. 


%%%%%%%%%%%%%%%%%%%%%%%%%%%%%%%%%%%%%%%%%%%%%%%%%%%%%%%%%%%%%%%%%%%%%%%%%%%%%%%%
\subsection{Limitations}
Survey research 
provides data on a wide range of issues that people may be reluctant to 
disclose, including mental health disorders, personal medical issues, 
prescription medications, and illicit drug use. Responses to surveys may be 
biased to some degree, and it can be difficult to obtain reliable information 
about illicit or illegal behaviors, but the anonymity of survey response are 
designed to preserve confidentiality and can help to assure more accurate 
disclosures. 


A limitation of the study is that only a small proportion of the sample 
reported having used or misused prescription opioids. It is possible that the 
level of opioid use in the sample was not representative of opioid use in the 
general population. According to the Centers for Disease Control, the rate
of heroin use in the general population of adults is 2.6 percent, whereas
the rate of heroin use in the NSDUH-2015 sample was 1.6 percent \cite{cdc16}. 
Obtaining reliable information about medication consumption can be difficult 
based on self-reports. Survey data can be biased by under-reporting or by 
minimizing reports of illicit substance use. People may also be reluctant to 
disclose mental health issues or health problems (e.g., STDs, HIV, suicide 
attempts). Another consideration is that, owing to the nature of the methods 
used in the study, it was not possible to determine the direction of the 
relationship between prescription opioid use and heroin use. Individuals who 
misuse or abuse prescription opioid medications may turn to synthetic opioids 
or heroin as cheaper, more readily available than prescription medications. 
Alternatively, individuals who used heroin may be inclined to abuse 
prescription opioids based on availability. Past research has shown that drug 
dosage and medication type play a significant role in opioid misuse: Treatment 
with high daily dose opioids (e.g., more than 120 mg/ day) and short-acting 
schedule II opioids increases the risk of misuse, abuse, and drug overdose 
\cite{sullivan10}. The results of the present analysis showed that demographic 
characteristics played a relatively minor role compared to the use of illicit 
drugs or tranquilizers. 

%%%%%%%%%%%%%%%%%%%%%%%%%%%%%%%%%%%%%%%%%%%%%%%%%%%%%%%%%%%%%%%%%%%%%%%%%%%%%%%%
\section{Conclusion}

A general conclusion is that people who reported misusing prescription opioids 
were also likely to have received substance treatment. More than any other 
demographic features, a history of prescription medication use, or illicit drug 
use, both seemed to contribute highly to the abuse of pain reliever medications, 
particularly for those who reported using tranquilizers, heroin, cocaine, or 
amphetamines. The findings suggest that the opioid crisis may be driven by the 
widespread availability of prescription medications. Even for people with no 
previous history of mental health issues, exposure to highly addictive opioid 
medications puts people at risk for drug dependency and addiction. As mentioned 
in the introduction, even for those who have been in drug treatment programs, 
a lack of continuity in treatment can leave many people in recovery at risk 
for relapse or possible overdose as they are released back into environments 
associated with their drug use. The sharp increase in overdose deaths in the 
U.S. due to synthetic opioids (other than methadone) has coincided with the 
increased availability of illicitly manufactured fentanyl \cite{nida17}. 
Because the dosage levels and potency of illicit opioids are largely unknown, 
there is greater risk of drug overdose death. Recent findings suggest the 
opioid overdose epidemic is getting worse, and requires urgent action to prevent 
opioid abuse, addiction, and death. The findings reported here seek to raise 
awareness about the risk factors for prescription opioid addiction for patients 
and health care providers in order to help reduce opioid overdose deaths. 

%%%%%%%%%%%%%%%%%%%%%%%%%%%%%%%%%%%%%%%%%%%%%%%%%%%%%%%%%%%%%%%%%%%%%%%%%%%%%%%%


\begin{table}
  \caption{Frequency Table of Mental Health Issues and Treatment NSDUH 2015
  \cite{samhsa16}}
  \label{tab:freq}
  \begin{tabular}{cccccc}
    \toprule
    Age Group & 12-17& 18-25& 26-34& 35-49& 50+\\
    \midrule
    In Hospital Overnight& 730& 1149& 821& 890& 1173 \\
    Adult Depression& 0& 2413& 1395& 1766& 967 \\
    \midrule
    Mental Health Treatment& & & & & \\
    \midrule
    Private Therapist& 0& 592& 434& 554& 311 \\
    Treatment Gap*& 469& 931& 321& 239& 90 \\
    \bottomrule
  \end{tabular}
\end{table}
%%%%%%%%%%%%%%%%%%%%%%%%%%%%%%%%%%%%%%%%%%%%%%%%%%%%%%%%%%%%%%%%%%%%%%%%%%%%%%%%
ACM-Reference-Format
\bibliographystyle{unsrt}
\bibliography{report} 


\end{document}
